
%%%%%%%%%%%%%%%%%%%%% %%%%%%%%%%%%%%%%%%%%% %%%%%%%%%%%%%%%%%%%%% 
% JNDとPSEの統計分析結果(Calm)
%%%%%%%%%%%%%%%%%%%%%% %%%%%%%%%%%%%%%%%%%%% %%%%%%%%%%%%%%%%%%%%% %%%%%%%%%%%%%%%%%%%%% 
\begin{figure}[t]
  % \vspace {-20pt}
  
  
  \begin{tabular}{ccc}
  
  \begin{minipage} {0.32\hsize}
  \centering
  \includegraphics[ width = 1\columnwidth]{Figure/ExpCalm/Fig_YNHEld_JND_IgnoreCue.eps }
  \end{minipage}&
  
  \begin{minipage} {0.32\hsize}
  \centering
  \includegraphics [ width = 1\columnwidth]{Figure/ExpCalm/Fig_YNHEld_PSE_IgnoreCue.eps }
  \end{minipage}&

  \begin{minipage} {0.32\hsize}
  \centering
  \includegraphics [ width = 1\columnwidth]{Figure/ExpCalm/Fig2c_CalEmoYNHEld_JNDNHHL_Mean4_Thrsh22dB.eps }
  \end{minipage}
  

  \end{tabular}
  
  \vspace {-6pt}
  \caption{実験条件(YNH-Unpro,YNH-80yr,Older-Unpro)ごとにおけるJND(a)とPSE(b)の参加者全体の平均値と、95\%信頼区間。
            点線はJND,PSEそれぞれの全条件の平均値を示す。
            (c)はYNH-Unpro、高齢健聴群(ONH)、高齢者難聴群(OHL)のJNDの平均値と95\%信頼区間。
            各3組のバーグラフは感情対に対応。落着き (Calm, C)。
            Tukey HSDの多重比較検定(有意水準 5\%)の結果も示す。}
  
  \label{fig:JNDPSE_Calm}

  % \vspace {-12pt}
  \end{figure}
  %%%%%%%%%%%%%%%%%%%%%% %%%%%%%%%%%%%%%%%%%%% %%%%%%%%%%%%%%%%%%%%% %%%%%%%%%%%%%%%%%%%%% 


