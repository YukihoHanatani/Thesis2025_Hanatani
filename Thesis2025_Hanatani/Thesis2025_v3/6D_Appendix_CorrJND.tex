%%%%%%%%%%%%%%%%%%%%%%%%%%%%%%%%%%%%%%%%%%%%%%%%%%%%%%%%%%%%%%%%
\chapter{JNDの相関分析}
\label{sec:CorrJND}
%%%%%%%%%%%%%%%%%%%%%%%%%%%%%%%%%%%%%%%%%%%%%%%%%%%%%%%%%%%%%%%% 
弁別実験を通して、高齢者は健聴・難聴に関わらず若年健聴者よりもJNDが大きく、弁別精度が下がることがわかった。
ここでは、聴力レベル以外の要因についてもJNDとの関係性があるかどうかを調査した結果を報告する。
結論として、JNDとさまざまな要素との相関関係を調査したが、ほぼ全てにおいて強い相関は見られなかった。


% 怒り・悲しみ・喜び実験参加者には、本研究室で同時期に行なっていたTMTF測定に参加してもらったため、TMTFとの相関も調べた。
% また、落着き実験参加者には聴覚の時間微細構造の感度を測定するTFS1実験を実施した。
% その結果との相関を報告する。

% ------------------------------
\section{年齢・平均聴力レベルとの相関}
% ------------------------------
まず、結果全体の傾向を把握するために、怒り・悲しみ・喜び実験、落着き実験結果におけるJNDと年齢、平均聴力レベルの相関を分析した。
結果をそれぞれ図\ref{fig:CorrAge}、図\ref{fig:CorrAud}に示す。 

図\ref{fig:CorrAge}には、JNDと年齢について若年健聴者Unproと高齢者結果を手掛語別にプロットしている。
また、それぞれの聴取者群結果と聴取者全体の結果の回帰直線を描画した。
相関分析の結果、喜び--怒り対で中程度の正の相関があった(p $<$ 0.01, r = 0.68)。
この結果は、高齢者において喜び--怒り対の判断が難しかったことと整合性がある。
怒り--悲しみ対、悲しみ--喜び対については相関は見られなかった。
また、怒り--落着き対(p $<$ 0.01, r = 0.51)、悲しみ--落着き対(p $<$ 0.01, r = 0.39)、喜び--落着き対(p $<$ 0.01, r = 0.57)、で正の相関が見られたが、
JNDが40\%以上の値を外れ値とみなすと相関はほとんどないと言えるだろう。



%%%%%%%%%%%%%%%%%%%%% %%%%%%%%%%%%%%%%%%%%% %%%%%%%%%%%%%%%%%%%%% %%%%%%%%%%%%%%%%%%%%% %%%%%%%%%%%%%%%%%%%%%% %%%%%%%%%%%%%%%%%%%%% %%%%%%%%%%%%%%%%%%%%% %%%%%%%%%%%%%%%%%%%%% 
\begin{figure}[h]

  % \begin{center}

  % \vspace {-20pt}
  \begin{tabular}{ccc}
    
    \begin{minipage} {0.31\hsize}
    \centering
    \includegraphics[ width = 1\columnwidth]{Figure/Appendix/6D/Fig_Scat_Age-JND_sad-ang.eps }
    \end{minipage}&
    
    \begin{minipage} {0.31\hsize}
    \centering
    \includegraphics[ width = 1\columnwidth]{Figure/Appendix/6D/Fig_Scat_Age-JND_hap-sad.eps }
    \end{minipage} &
    
    \begin{minipage} {0.31\hsize}
    \centering
    \includegraphics [ width = 1\columnwidth]{Figure/Appendix/6D/Fig_Scat_Age-JND_ang-hap.eps }
    \end{minipage} 
    
  \\  %% 改行  %%%%%%%%%%%%%%%%%%%%% 

    \begin{minipage} {0.31\hsize}
    \centering
    \includegraphics[ width = 1\columnwidth]{Figure/Appendix/6D/Fig_Scat_Age-JND_cal-ang.eps }
    \end{minipage}&
    
    \begin{minipage} {0.31\hsize}
    \centering
    \includegraphics [ width = 1\columnwidth]{Figure/Appendix/6D/Fig_Scat_Age-JND_cal-sad.eps }
    \end{minipage} &
    
    \begin{minipage} {0.31\hsize}
    \centering
    \includegraphics [ width = 1\columnwidth]{Figure/Appendix/6D/Fig_Scat_Age-JND_cal-hap.eps }
    \end{minipage} 

  \end{tabular}

  \vspace {-6pt}
  \caption{全実験参加者のJNDと年齢との相関。左上から、怒--悲、悲–-喜、喜-–怒、怒--落、悲–-落、喜-–落実験結果。
           縦軸はJND、横軸は実験実施時の年齢。
           ○:若年健聴者(Unpro)、◇:高齢者。手掛語(赤:怒り、青:悲しみ、 緑:喜び、 黄:落着き)。 
           回帰直線を、赤線:若年健聴者(Unpro)、青線:高齢者、ピンク線:若年健聴者(Unpro)+高齢者で示す。
          }

  \label{fig:CorrAge}

  \vspace {-12pt}
\end{figure}
%%%%%%%%%%%%%%%%%%%%%% %%%%%%%%%%%%%%%%%%%%% %%%%%%%%%%%%%%%%%%%%% %%%%%%%%%%%%%%%%%%%%% %%%%%%%%%%%%%%%%%%%%%% %%%%%%%%%%%%%%%%%%%%% %%%%%%%%%%%%%%%%%%%%% %%%%%%%%%%%%%%%%%%%%% 


図\ref{fig:CorrAud}には、JNDと平均聴力レベルについて若年健聴者Unpro、若年健聴者80yr、高齢者結果を手掛語別にプロットしている。
% また、それぞれの聴取者群結果と聴取者全体の結果の回帰直線を描画した。
ここでは、若年健聴者80yrの平均聴力レベルを、若年健聴者の平均聴力レベルに模擬難聴処理で模擬した80歳の平均聴力レベル\cite{tsuiki2002nihon}を加算した値に設定した。
相関分析の結果、喜び--怒り対で中程度の正の相関があった(p $<$ 0.01, r = 0.36)。
若年健聴者80yrの結果を除いた場合、強い正の相関が見られた(p $<$ 0.01, r = 0.77)。
喜び--怒り対では、感情弁別に年齢が少なからず影響している可能性がある。
また、怒り--落着き対(p $<$ 0.05, r = 0.26)、悲しみ--落着き対(p $<$ 0.05, r = 0.27)、喜び--落着き対(p $<$ 0.05, r = 0.25)、で弱い正の相関が見られたが、
やはり外れ値を除くと相関はほとんどないと言えるだろう。
全体的に見ると、若年健聴者80yrより高齢者の方が平均聴力レベルが小さい傾向にあるが、JNDのばらつきが大きい。
一方で、若年健聴者80yrは若年健聴者Unproとほぼ同程度の高さに位置しており、つまり模擬難聴処理の有無でJNDに大きな変化はなかったことがわかる。
したがって、聴力レベル以外の何かしらの要因が影響していると考えられる。




%%%%%%%%%%%%%%%%%%%%% %%%%%%%%%%%%%%%%%%%%% %%%%%%%%%%%%%%%%%%%%% %%%%%%%%%%%%%%%%%%%%% %%%%%%%%%%%%%%%%%%%%%% %%%%%%%%%%%%%%%%%%%%% %%%%%%%%%%%%%%%%%%%%% %%%%%%%%%%%%%%%%%%%%% 
\begin{figure}[h]

  % \begin{center}

  \vspace {10pt}
  \begin{tabular}{ccc}
    
    \begin{minipage} {0.31\hsize}
    \centering
    \includegraphics[ width = 1\columnwidth]{Figure/Appendix/6D/Fig_Scat_Agram-JND_Mean4_sad-ang.eps }
    \end{minipage}&
    
    \begin{minipage} {0.31\hsize}
    \centering
    \includegraphics[ width = 1\columnwidth]{Figure/Appendix/6D/Fig_Scat_Agram-JND_Mean4_hap-sad.eps }
    \end{minipage} &
    
    \begin{minipage} {0.31\hsize}
    \centering
    \includegraphics [ width = 1\columnwidth]{Figure/Appendix/6D/Fig_Scat_Agram-JND_Mean4_ang-hap.eps }
    \end{minipage} 
    
  \\  %% 改行  %%%%%%%%%%%%%%%%%%%%% 

    \begin{minipage} {0.31\hsize}
    \centering
    \includegraphics[ width = 1\columnwidth]{Figure/Appendix/6D/Fig_Scat_Agram-JND_Mean4_cal-ang.eps }
    \end{minipage}&
    
    \begin{minipage} {0.31\hsize}
    \centering
    \includegraphics [ width = 1\columnwidth]{Figure/Appendix/6D/Fig_Scat_Agram-JND_Mean4_cal-sad.eps }
    \end{minipage} &
    
    \begin{minipage} {0.31\hsize}
    \centering
    \includegraphics [ width = 1\columnwidth]{Figure/Appendix/6D/Fig_Scat_Agram-JND_Mean4_cal-hap.eps }
    \end{minipage} 

  \end{tabular}

  \vspace {-6pt}
  \caption{全実験参加者のJNDと平均聴力レベルとの相関。▲:若年健聴者(80yr)。黒線:若年健聴者(Unpro)+若年健聴者(80yr)+高齢者の回帰直線。}

  \label{fig:CorrAud}

  \vspace {-12pt}
\end{figure}
%%%%%%%%%%%%%%%%%%%%%% %%%%%%%%%%%%%%%%%%%%% %%%%%%%%%%%%%%%%%%%%% %%%%%%%%%%%%%%%%%%%%% %%%%%%%%%%%%%%%%%%%%%% %%%%%%%%%%%%%%%%%%%%% %%%%%%%%%%%%%%%%%%%%% %%%%%%%%%%%%%%%%%%%%% 



% ------------------------------
\section{TMTF測定結果との相関}
% ------------------------------
怒り・悲しみ・喜び実験の高齢参加者12名のうち11名は、本研究室で同時期に行なっていたTMTF測定に参加していたため、その測定値との相関も調べた。
%健聴者はとったけど分析できてない %FHYは取ってない
このTMTF測定では、森本らの提案した2点法を用いている\cite{morimoto2019Two-PointTMTF}。
変調度の閾値($L_{ps}$)[dB]との結果を図\ref{fig:JNDTMTF}に示す。
図から、変調度の閾値が大きくなるほど、JNDも大きくなる傾向が見られる。
相関分析の結果、中程度の正の相関があった(p $<$ 0.05, r = 0.34)。
時間分解能の低下は感情弁別精度の低下の要因の1つとして考えられるかもしれない。


% ---------------------------------------
\begin{figure}[htbp]
  \vspace{40pt}
  \centering
  \includegraphics[width=0.6\hsize]{Figure/Appendix/6D/Fig_CorrJNDSpIntel_TMTFLps_All.eps}
  \caption{怒り・悲しみ・喜び実験参加の高齢者11名のJNDと、TMTFの変調度の閾値($L_{ps}$)との相関。黒線は回帰直線。}
  \label{fig:JNDTMTF}
\end{figure}
% --------------------------------------------




% ------------------------------
\section{TFS測定結果との相関}
% ------------------------------
落着き実験参加の高齢者のうち、JNDが外れ値だった1名を除いた11名には、時間微細構造(TFS)への感度を測定する実験を行なった。
ここでは、Sekらの測定方法\cite{Sek2022Guide,Sek2012TFS1}に則って実施された。
純音の高さの変動に対する検知閾と、複合音のピッチ変動に対する検知閾について、それぞれJNDとの相関を調べた。
結果を図\ref{fig:JNDTFS}に示す。
なお、複数の参加者においてスケールアウトしたため、測定できなかった値を欠損値として扱っている。
%F0=200Hzの複合音

純音テストでは、怒り--落着き対(p $<$ 0.05, r = -0.33)と喜び--落着き対(p $<$ 0.05, r = 0.04)で弱い相関が見られた。
これらの相関は前者が負、後者が正の相関を示している。
この結果を見るに、これだけでJNDとの関係性を説明することは難しいように思われる。
複合音テストでは、どの感情対においても相関は見られなかった。
全体的に、高齢者にとってこの測定方法は難しく、測定における手順や内容を完全に理解できていない可能性が大いにある。
実際に、半数の参加者が特に複合音テストで片耳ないし両耳でスケールアウトした。
高齢者にもわかりやすい測定手法を用いて、再度検討する必要があるだろう。



\begin{figure}[h]

  % \begin{center}

  \vspace {-20pt}
  \begin{tabular}{ccc}
    
    \begin{minipage} {0.31\hsize}
    \centering
    \includegraphics[ width = 1\columnwidth]{Figure/Appendix/6D/Fig_Scat_TFS1BothEarPureTone-JND_cal-ang.eps }
      %スペース
    \end{minipage}&
    
    \begin{minipage} {0.31\hsize}
    \centering
    \includegraphics [ width = 1\columnwidth]{Figure/Appendix/6D/Fig_Scat_TFS1BothEarPureTone-JND_cal-sad.eps}
    \subcaption{純音}
    \end{minipage} &
    
    \begin{minipage} {0.31\hsize}
    \centering
    \includegraphics [ width = 1\columnwidth]{Figure/Appendix/6D/Fig_Scat_TFS1BothEarPureTone-JND_cal-hap.eps}
     
    \end{minipage} 
    
  \\  %% 改行  %%%%%%%%%%%%%%%%%%%%% 

    \begin{minipage} {0.31\hsize}
    \centering
    \includegraphics[ width = 1\columnwidth]{Figure/Appendix/6D/Fig_Scat_TFS1BothEarHarmonicTone-JND_cal-ang.eps }
      %スペース
    \end{minipage}&
    
    \begin{minipage} {0.31\hsize}
    \centering
    \includegraphics [ width = 1\columnwidth]{Figure/Appendix/6D/Fig_Scat_TFS1BothEarHarmonicTone-JND_cal-sad.eps }
    \subcaption{調波複合音}
    \end{minipage} &
    
    \begin{minipage} {0.31\hsize}
    \centering
    \includegraphics [ width = 1\columnwidth]{Figure/Appendix/6D/Fig_Scat_TFS1BothEarHarmonicTone-JND_cal-hap.eps }
     
    \end{minipage} 

  \end{tabular}

  \vspace {-6pt}
  \caption{落着き実験参加の高齢者11名のJNDとTFS測定結果。○:右耳、×:左耳。縦軸はJND。
            横軸は、(a)1kHzの純音のピッチ変動の閾値、 (b)調波複合音のピッチ変動の閾値。青線は回帰直線を示す。}

  \label{fig:JNDTFS}

  \vspace {-12pt}
\end{figure}
%%%%%%%%%%%%%%%%%%%%%% %%%%%%%%%%%%%%%%%%%%% %%%%%%%%%%%%%%%%%%%%% %%%%%%%%%%%%%%%%%%%%% %%%%%%%%%%%%%%%%%%%%%% %%%%%%%%%%%%%%%%%%%%% %%%%%%%%%%%%%%%%%%%%% %%%%%%%%%%%%%%%%%%%%% 











































