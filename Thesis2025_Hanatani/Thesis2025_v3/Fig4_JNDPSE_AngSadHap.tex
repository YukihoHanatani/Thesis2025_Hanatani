
%%%%%%%%%%%%%%%%%%%%% %%%%%%%%%%%%%%%%%%%%% %%%%%%%%%%%%%%%%%%%%% 
% JNDとPSEの統計分析結果(AngSadHap)
%%%%%%%%%%%%%%%%%%%%%% %%%%%%%%%%%%%%%%%%%%% %%%%%%%%%%%%%%%%%%%%% %%%%%%%%%%%%%%%%%%%%% 
\begin{figure}[t]
  % \vspace {-20pt}
  
  
  \begin{tabular}{ccc}
  
  \begin{minipage} {0.32\hsize}
  \centering
  \includegraphics[ width = 1\columnwidth]{Figure/ExpAngHapSad/Fig_YNHEld23_JND_IgnoreCue.eps }
  \end{minipage}&
  
  \begin{minipage} {0.32\hsize}
  \centering
  \includegraphics [ width = 1\columnwidth]{Figure/ExpAngHapSad/Fig_YNHEld23_PSE_IgnoreCue.eps }
  \end{minipage}&

  \begin{minipage} {0.32\hsize}
  \centering
  \includegraphics [ width = 1\columnwidth]{Figure/ExpAngHapSad/Fig2c_CalEmoYNHEld_JNDNHHL_Mean4_Thrsh22dB.eps }
  \end{minipage}
  

  \end{tabular}
  
  \vspace {-6pt}
  \caption{実験条件(YNH-Unpro,YNH-80yr,Older-Unpro)ごとにおけるJND(a)とPSE(b)の参加者全体の平均値と、95\%信頼区間。
            点線はJND,PSEそれぞれの全条件の平均値を示す。
            (c)はYNH-Unpro、高齢健聴群(ONH)、高齢者難聴群(OHL)のJNDの平均値と95\%信頼区間。
            各3組のバーグラフは感情対に対応。怒り (Angry, A), 悲しみ (Sad, S), 喜び (Happy,H) の2つの組み合わせで表示。
            Tukey HSDの多重比較検定(有意水準 5\%)の結果も示す。}
  
  \label{fig:JNDPSE_AngSadHap}

  \vspace {-12pt}
  \end{figure}
  %%%%%%%%%%%%%%%%%%%%%% %%%%%%%%%%%%%%%%%%%%% %%%%%%%%%%%%%%%%%%%%% %%%%%%%%%%%%%%%%%%%%% 


%   実験条件(YNH-Unpro、YNH-80yr、Older-Unpro)における参加者全体のJND(a)とPSE(b)の平均値と95%信頼区間(%)。
%   A:Ang、S:Sad、H:Hap。パネル(c)は、YNH-Unpro、高齢NH(ONH)、高齢HL(OHL)を比較するためのJNDの平均と95%CIを示す。
%   Tukey's HSD検定はα=0.05で行った。パネル(a)では、アスタリスク(*)を付けた高齢参加者のHap-Angペアと他のペアとの間に有意差があった。
%   破線は、最後の2本の棒を除くすべての条件の平均を表す。パネル(b)では、アスタリスク(*)のついたペアは他のペアと有意差があった。
%   パネル(c)では、古いNHのHap-Angペアとも有意差があった。

%   実験条件(YNH-Unpro,YNH-80yr,Older-Unpro)ごとの JND(a)と PSE(b). 各 3 組のデータは感情対に対応. 落着き (Calm,
% C), 怒り (Angry, A), 悲しみ (Sad, S), 喜び (Happy,H) の 2 つの組み合わせで表示. 25,50,75 パーセンタイル箱ひげ図で表示. 参加者の
% JND/PSE を手掛語(怒り*, 悲しみ△, 喜び◯, 落着き■)ごとに示す. Tukey HSD の多重比較検定(有意水準 5%) の結果も示す

