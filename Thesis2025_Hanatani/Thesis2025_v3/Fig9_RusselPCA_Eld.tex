%%%%%%%%%%%%%%%%%%%%%% %%%%%%%%%%%%%%%%%%%%% %%%%%%%%%%%%%%%%%%%%% 
% 主成分分析結果・Russel  Calm


% ----------------------------------%
% PCA Eld  and YNH7

 \begin{figure*}[t]
  % \vspace{-18pt}
  
    \begin{tabular}{cc}
    % \hspace{-32pt}
    \begin{minipage} {0.5\hsize}
    \centering
    \includegraphics[ width = 1\columnwidth]{Figure/ExpCalm//FigPCA3dYNH7_CumAll_cal2_scatter_SndTriangle_Eng.eps}
    \subcaption{若年健聴者7名の結果}
    \end{minipage}&
    \hspace{-22pt}
    
    \begin{minipage} {0.5\hsize}
    \centering
    \includegraphics [ width = 1\columnwidth]{Figure/ExpCalm/FigPCA3dEld_CumAll_cal2_scatter_SndTriangle_Eng_RotateC.eps} 
    \subcaption{高齢者11名の結果}
    \end{minipage} 
  
  
  \end{tabular}
  
  \caption{実験者3名を含む若年健聴者7名と高齢者11名による感情尺度評定の主成分分析 (PCA) 結果。 図\ref{fig:PCA-Russel_Calm}の若年健聴者(実験者3名)結果に対応。}
  \label{fig:PCA_Eld-YNHNew}
\end{figure*}


% % ---------------------------------------
% \begin{figure}[h]
%   \vspace{40pt}
%   \hspace{20pt}
%   \centering
%   \includegraphics[width=0.8\hsize]{Figure/ExpCalm/FigPCA3d_CumAll_cal2_scatter_SndTriangle_Eng.eps}
%   \caption{感情尺度評定の主成分分析(PCA)結果。各点(x)は、怒り,悲しみ,喜び,落着きの評価を行って得られたPCAスコアを示す。
%             "Ang"怒り,"Sad"悲しみ,"Hap"喜び,"Cal"落着き。 見やすさのため3次元で描画している。}
% \end{figure}
% % --------------------------------------------



