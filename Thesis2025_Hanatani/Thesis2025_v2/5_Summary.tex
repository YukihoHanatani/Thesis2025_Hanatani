%%%%%%%%%%%%%%%%%%%%%%%%%%%%%%%%%%%%%%%%%%%%%%%%%%%%%%%%%%%%%%%%%%%%%%%%%%%%%%%%%%
%%%%%%%%%%%%%%%%%%%%%%%%%%%%%%%%%%%%%%%%%%%%%%%%%%%%%%%%%%%%%%%%%%%%%%%%%%%%%%%%%%
\chapter{総括}
%%%%%%%%%%%%%%%%%%%%%%%%%%%%%%%%%%%%%%%%%%%%%%%%%%%%%%%%%%%%%%%%%%%%%%%%%%%%%%%%%%
%%%%%%%%%%%%%%%%%%%%%%%%%%%%%%%%%%%%%%%%%%%%%%%%%%%%%%%%%%%%%%%%%%%%%%%%%%%%%%%%%%
%%%%%%%%%%%%%%%%%%%%%%%%%%%%%%%%%%
\section{本論文のまとめ}
\label{sec:Summary}
%%%%%%%%%%%%%%%%%%%%%%%%%%%%%%%%%%
高齢者の感情知覚特性について新たな知見を得ることを目的に、若年健聴者と高齢者を対象に2種類の感情弁別実験を実施した。
難聴の原因について、その切り分けは困難であるため、まずは聴覚末梢系の機能低下のみが感情知覚にどの程度影響するのか調査することにした。
そのために、若年健聴者実験では聴覚末梢系の機能低下だけを模擬する模擬難聴システムを導入した。
若年健聴者が通常音声を聞いた場合・模擬難聴音声を聞いた場合・高齢者が同じ通常音声を聞いた場合の3条件の比較を行った。

第\ref{chap:ExpAngHapSad}章では、怒り・悲しみ・喜び音声間での弁別精度を調査した。
その結果、若年健聴者実験では模擬難聴処理の有無でJND、PSEに有意差は無く、
聴覚末梢系の機能低下だけは感情知覚に影響しない可能性が示唆された。
また、高齢者にとっては「喜び–怒り」間の弁別だけが著しく難しいことがわかった。
この結果から、ラッセルの感情円環モデルを用いて「覚醒–沈静の度合いが同程度の場合、快–不快の判断が難しい」
という作業仮説を立てた。

第\ref{chap:ExpCalm}章では、この作業仮説の立証のために「快」かつ「沈静」の感情である「落着き」を含めた実験を実施した。
その結果、覚醒度が同程度の「悲しみ-落着き」間の弁別は他の感情対と比較して容易であることがわかった。
すなわち、作業仮説は反証された。
また、若年健聴者実験では模擬難聴処理の有無でJND、PSEに有意差は無く、これは怒り・悲しみ・喜び実験の結果と整合性がある。
高齢者に関して、健聴・難聴に関わらず若年健聴者よりも弁別精度が下がることがわかった。

これらの実験結果を比較すると、「悲しみ」音声を含む感情対でJNDが小さく、「悲しみ」音声が演技的で弁別しやすかった可能性が考えられる。
しかしながら、弁別のしやすさの要因がこれだけによるものなのかはわからない。
JNDと年齢・平均聴力レベル・時間変調伝達関数(TMTF)・時間微細構造(TFS)の感度との相関分析も行ったが、強い相関は見られなかった。
詳細は付録\ref{sec:CorrJND}にて述べる。



%Mod
% しかし、なぜ喜び-怒り対だけが有意に弁別しにくいのかの原因はわかっていない。少なくとも健聴者にとっては両者の違いは明確なので、直感に反する。
% 韻律情報のダイナミックスが類似している可能性を考え、基本周波数の分散の比較も予備的に行ったが、これだけでは十分な説明はできそうになかった。


%
%悲しみ音声演技的 → 反対に言えば、「演技的」かどうかは聴力レベルや年齢に関わらず判断できる!わかる!
% → この「演技的」と言われる所以・要因が解明できれば、その特性を音声合成に反映させることで、高齢者にも伝わる「悲しみ音声」が合成できる

%相関
% 年齢・聴力・TMTFとの相関結果

%音声の基本周波数


%%%%%%%%%%%%%%%%%%%%%%%%%%%%%%%%%%
\section{今後の課題と展望}
\label{sec:Challenges}
%%%%%%%%%%%%%%%%%%%%%%%%%%%%%%%%%%
本研究では、音声データを男性1名が発声した10単語を用いた。
今回得られた結果の妥当性を高めるためには、女性を含めた複数話者の複数の単語を用いて実験する必要がある。
また、今回用いたような感情を意図的に想起して発声された音声ではなく、自然な会話の中での感情音声を用いることも必要であろう。
% そのための感情音声の収録も行った(付録\textcolor{red}{****})。
若年健聴者・高齢者のともに「悲しみ」音声を含む弁別が容易であることが示唆されたが、これが今回用いた音声の話者特性によるものかを
検証する必要がある。
この話者の「悲しみ」音声に関して、他の感情に比べて演技度が高いという主観評価実験結果が得られた。
このことは、感情音声が演技的であるかどうかは、年齢や聴力レベルに関わらず判別できるということである。
演技的と感じられる要因となっている音声の特徴量が解明できれば、それらを音声合成に反映させることで
高齢難聴者にも伝わる感情音声の合成に役立てることができるだろう。

本実験では、高齢者にとって「覚醒–沈静の度合いが同程度の場合、快–不快の判断が難しい」ことは反証されたが、
「覚醒度が高い場合、快–不快の判断が難しい」可能性も考えられる。
ラッセルの感情円環モデル上では、基本6感情のうち「怒り」「喜び」のほかに「嫌悪」「驚き」「恐怖」などの感情が覚醒度が高い。
これらの感情間でも弁別実験を行ってみることで、さらなる知見が期待できる。
いずれにせよ、ラッセルの感情円環モデルは心理空間を構成する1つの仮説にすぎず、音声の物理的特徴量や聴覚的な内部表現との関連性は明確ではない。
高齢者の感情知覚特性とそれに影響する要因の解明へのアプローチとして、両者の関係を聴覚モデルを通して分析することも重要であろう。



%音声データの変更 → 複数話者・女声・他の単語 → そのための収録も行った, 別の悲しみ音声はわかる?

%「演技的」な要素とは? 基本周波数?

%ラッセルの円環モデルの妥当性、聴覚モデルとの関係性

% 高齢者は「覚醒」度の高い感情間で難しい? → 嫌悪・驚き・恐怖を含めた実験。恐怖のみ本実験音声とはPCA頭上で一致しなかった






% ASJHJan2024
% 今回得られた結果は、高齢難 聴における末梢系レベルの機能低下は、感情知覚への影響が小 さいということを示唆している。
% ある意味、「あたりまえ」と捉 えられるかもしれないが、このように明確に示された研究は報告されていないと思われる。

% さらに興味深いのは、感情の種類により、高齢者と若年健聴者で知覚特性が異なることが示唆されたことにある。
% 判断が難しかった怒りと喜びは、図 2 のラッセルの円環モデルにおいて、 「快-不快」軸上で反対側に位置する。
% 一方、「覚醒-沈静」軸上では原点より上で覚醒レベルが同程度の位置にある。
% これに対し、 怒りと悲しみ・悲しみと喜びに関しては、「覚醒-沈静」軸上で は反対側に位置している。
% これらの間の刺激連続体においては 健聴者と同程度の弁別ができた。
% これらのことから、高齢者は「覚醒-沈静」の度合いが同程度の場合、「快-不快」の判断が難 しい可能性がある。

% 直感的には「快-不快」の判断の方が容易に 思えるので、この結果の要因が何かがわからない。
% このことを 確かめるためには、第 4 象限にある Relaxed/At Ease (「安心」) を加えて、安心-悲しみ対や安心-喜び対の実験を行っても良い かもしれない。
% ただし、今回の実験で用いたのが男性一名の限 られた単語数の音声であったために、このような結果になった 可能性も否定できない。

% 複数話者の実験も必要であろう。
% この 場合でも、今回用いた模擬難聴と音声モーフィングを用いた独 自の手法を用いることができ、さらなる知見が期待できる。
% いずれにせよラッセルの円環モデルは心理空間を構成する一 つの仮説にすぎず、音声の物理的特徴量や聴覚的な内部表現と の関連性は明確ではない。
%  この両者の関係を聴覚モデルを通し て分析することにより、高齢者の感情知覚特性とそれに影響す る要因の解明へアプローチすることも重要であろう。

%Scirep2024
%模擬難聴の影響について
% その結果、YNHの加齢に伴う高音域のシミュレーションは、すべての感情ペアにおいて感情弁別に影響を及ぼさなかった。
% この結果は、補聴器が感情弁別能力を向上させなかった9,12や、HLシミュレーションがNHの怒り認知に影響を与えなかった14という過去の知見と一致する。
% これらの結果を総合すると、低周波数に現れる音声の韻律情報が感情認知に重要であることが示唆される3,6,10。
% 感情認識実験から得られた知見は、今回の3つの感情識別研究によって確認された。
%  HLシミュレーションの影響を受けない感情があるかどうかを明らかにするためには、他の感情を用いた実験を行う必要がある。

% 高齢者ang-hapについて
%  さらに興味深いことに、年配の参加者はAngとHapの識別が困難であったが、図2aに示すように、他のペアではYNHの識別能力に近かった。
% 図2cは、AngとHapの識別がONHでは難しく、OHLではより難しいことを示している。
% われわれの知る限り、このような観察は報告されていないようである。
% 若年者と中年者の感情弁別/カテゴリー知覚実験に関する論文4がある。
%  そこでは、恐怖-幸福、幸福-悲しみのペアでは、幸福の識別が相対的に悪くなることが報告されている。残念ながら、HLや年齢の異なる条件間での比較は行われていない。
% そのため、似たようなことを言うのは難しいが、幸福の根底にある認識について何らかの洞察を与えてくれるかもしれない。
% また、Ang-Hapペアにおける年齢効果を、若年から高齢まで幅広い年齢のNHおよびHL参加者を用いた実験で、4つの感情(嬉しい、悲しい、怖い、怒っている)すべてに年齢効果が観察された感情認知研究11の結果からのみ解釈することは難しいようだ。
%  また、年齢とともに感情認知の精度が低下することも報告されている7,10,13。
%  しかし、Hap-Angペアと他のペアとの間の非対称性を、これらの観察結果だけで説明するのは難しいように思われる。

% ↑に対する解釈、ラッセル
%  次に、カテゴリー知覚の結果を次元コア感情理論から解釈することを試みる。
% 図3aは、感情評価のPCAの結果を示しており、図1aに示した刺激セットを生成するために使用される(方法のセクションを参照)。
% Ang、Sad、Hapの感情の位置は、図1bに示すラッセルの円周モデルとほぼ一致している。
%  AngとHapは、年配の参加者にとって犯罪を判別するのが難しく、快/不快軸の反対側に位置している。
% 一方、高齢者とYNH参加者でほぼ同じJNDでAngやHapと弁別されたSadは、低い覚醒レベルにある。
% このことから、高齢者では、覚醒度が同程度の場合、快・不快の弁別が困難である可能性が示唆される。
% このことを確認するためには、第4象限の「穏やか」または「穏やか」と3つの感情(Sad、Hap、Ang)の組み合わせについて弁別実験を行うのがよいかもしれない。

% 聴覚モデルについて
% 新しい補聴器のアルゴリズムに関する2つ目の研究課題に答えるためには、結果を説明する聴覚モデルを開発することが不可欠です。
%  上記の仮説は、覚醒度に関連しうる韻律情報のダイナミクスが類似している場合、高齢者は感情の識別が困難であると言い換えることができる。
% 予備的な分析では、基本周波数Foの変動はAng語とHap語の間で同程度であり、Sad語のそれよりも有意に大きかった。
%  この分析だけでは聴覚系の機能障害を特定するには不十分であるため、高齢者の音声分析をシミュレートできる聴覚モデルを開発する必要がある。
% 例えば、聴覚フィルターバンクと変調フィルターバンクを備えたモデルがあり、HLと劣化した時間応答を導入することができる24,29,30。
% このモデルは、考えられる機能障害を理解し、感情知覚をサポートする効果的なアルゴリズムを開発するのに役立つだろう。

% 今後の課題
% この結果は直感に反する。というのも、快と不快の差は、少なくともNHの人においてはかなり明白であるように思われ、年齢が上がるにつれて変化するとは予想されなかったからである。
% その原因はまだ不明であり、加齢に伴う末梢HLではこの結果は説明できない。
%  今回の実験では、男性話者が発音した限られた単語しか使用していないため、より良い理解のためには、より多くの実験データを収集する必要がある。
%  より多くの男女の話者と異なる種類の刺激を用いた実験が必要である。
% また、Hap、Sad、Ang、その他の感情など、異なるペア間での識別実験も必要である。
% この場合、実験回数はnC2(nは対象感情の数)と組み合わせ論的に増加するので、感情の選択は慎重に行う必要がある。


% このため、弁別実験における判断も感情そのものではなく、その音声の特徴に基づいて行われた可能性がある。
% しかし、印象と特徴と言った抽象的な言葉への置き換えではなく、計算論的に解明できることが必要で、
% これこそ感情伝達の支援システムの構築の基盤となる。いずれにせよ、まだ十分解明できていない段階であり、
% 次の段階として男女複数話者の音声を用いて弁別実験を行い特性を把握する必要がある。
% そのための感情音声収集は行われており、今後の研究が期待される。

% この実験により、模擬難聴により末梢系機能の切り分けをした議論の可能性は示せたと考える。
% さらに、高齢者の感情知覚特性の一端を垣間見る良い機会となったことは確かであろう。
