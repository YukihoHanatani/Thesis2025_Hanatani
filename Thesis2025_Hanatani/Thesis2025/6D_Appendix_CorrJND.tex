%%%%%%%%%%%%%%%%%%%%%%%%%%%%%%%%%%%%%%%%%%%%%%%%%%%%%%%%%%%%%%%%
\chapter{JNDの相関分析}
\label{sec:CorrJND}
%%%%%%%%%%%%%%%%%%%%%%%%%%%%%%%%%%%%%%%%%%%%%%%%%%%%%%%%%%%%%%%% 
弁別実験を通して、高齢者は健聴・難聴に関わらず若年健聴者よりもJNDが大きく、弁別精度が下がることがわかった。
したがって、聴力レベル以外の要因がJNDとの関係性があるかもしれない。
そこで、JNDとさまざまな要素との相関関係を調査したが、全てにおいて強い相関は見られなかった。
それぞれの結果について報告する。


% 怒り・悲しみ・喜び実験参加者には、本研究室で同時期に行なっていたTMTF測定に参加してもらったため、TMTFとの相関も調べた。
% また、落着き実験参加者には聴覚の時間微細構造の感度を測定するTFS1実験を実施した。
% その結果との相関を報告する。

% ------------------------------
\section{年齢・平均聴力レベルとの相関}
% ------------------------------
まず、結果全体の傾向を把握するために、怒り・悲しみ・喜び実験、落着き実験結果におけるJNDと年齢、平均聴力レベルの相関を分析した。
結果をそれぞれ図\ref{fig:CorrAge}、図\ref{fig:CorrAud}に示す。 

 

%%%%%%%%%%%%%%%%%%%%% %%%%%%%%%%%%%%%%%%%%% %%%%%%%%%%%%%%%%%%%%% %%%%%%%%%%%%%%%%%%%%% %%%%%%%%%%%%%%%%%%%%%% %%%%%%%%%%%%%%%%%%%%% %%%%%%%%%%%%%%%%%%%%% %%%%%%%%%%%%%%%%%%%%% 
\begin{figure}[h]

  % \begin{center}

  % \vspace {-20pt}
  \begin{tabular}{ccc}
    
    \begin{minipage} {0.31\hsize}
    \centering
    \includegraphics[ width = 1\columnwidth]{Figure/Appendix/6D/Fig_Scat_Age-JND_sad-ang.eps }
    \end{minipage}&
    
    \begin{minipage} {0.31\hsize}
    \centering
    \includegraphics[ width = 1\columnwidth]{Figure/Appendix/6D/Fig_Scat_Age-JND_hap-sad.eps }
    \end{minipage} &
    
    \begin{minipage} {0.31\hsize}
    \centering
    \includegraphics [ width = 1\columnwidth]{Figure/Appendix/6D/Fig_Scat_Age-JND_ang-hap.eps }
    \end{minipage} 
    
  \\  %% 改行  %%%%%%%%%%%%%%%%%%%%% 

    \begin{minipage} {0.31\hsize}
    \centering
    \includegraphics[ width = 1\columnwidth]{Figure/Appendix/6D/Fig_Scat_Age-JND_cal-ang.eps }
    \end{minipage}&
    
    \begin{minipage} {0.31\hsize}
    \centering
    \includegraphics [ width = 1\columnwidth]{Figure/Appendix/6D/Fig_Scat_Age-JND_cal-sad.eps }
    \end{minipage} &
    
    \begin{minipage} {0.31\hsize}
    \centering
    \includegraphics [ width = 1\columnwidth]{Figure/Appendix/6D/Fig_Scat_Age-JND_cal-hap.eps }
    \end{minipage} 

  \end{tabular}

  \vspace {-6pt}
  \caption{全実験参加者のJNDと年齢との相関分析結果。左上から、怒--悲、悲–-喜、喜-–怒、怒--落、悲–-落、喜-–落実験結果。
           縦軸はJND、横軸は実験実施時の年齢。
           ◯:若年健聴者、♢:高齢者。手掛語(赤:怒り、青:悲しみ、 緑:喜び、 黄:落着き)。 
          %  回帰直線を、赤線:若年健聴者結果、青線:高齢者結果、ピンク線:全参加者結果で示す。
          }

  \label{fig:CorrAge}

  \vspace {-12pt}
\end{figure}
%%%%%%%%%%%%%%%%%%%%%% %%%%%%%%%%%%%%%%%%%%% %%%%%%%%%%%%%%%%%%%%% %%%%%%%%%%%%%%%%%%%%% %%%%%%%%%%%%%%%%%%%%%% %%%%%%%%%%%%%%%%%%%%% %%%%%%%%%%%%%%%%%%%%% %%%%%%%%%%%%%%%%%%%%% 

%%%%%%%%%%%%%%%%%%%%% %%%%%%%%%%%%%%%%%%%%% %%%%%%%%%%%%%%%%%%%%% %%%%%%%%%%%%%%%%%%%%% %%%%%%%%%%%%%%%%%%%%%% %%%%%%%%%%%%%%%%%%%%% %%%%%%%%%%%%%%%%%%%%% %%%%%%%%%%%%%%%%%%%%% 
\begin{figure}[h]

  % \begin{center}

  % \vspace {-20pt}
  \begin{tabular}{ccc}
    
    \begin{minipage} {0.31\hsize}
    \centering
    \includegraphics[ width = 1\columnwidth]{Figure/Appendix/6D/Fig_Scat_Agram-JND_Mean4_sad-ang.eps }
    \end{minipage}&
    
    \begin{minipage} {0.31\hsize}
    \centering
    \includegraphics[ width = 1\columnwidth]{Figure/Appendix/6D/Fig_Scat_Agram-JND_Mean4_hap-sad.eps }
    \end{minipage} &
    
    \begin{minipage} {0.31\hsize}
    \centering
    \includegraphics [ width = 1\columnwidth]{Figure/Appendix/6D/Fig_Scat_Agram-JND_Mean4_ang-hap.eps }
    \end{minipage} 
    
  \\  %% 改行  %%%%%%%%%%%%%%%%%%%%% 

    \begin{minipage} {0.31\hsize}
    \centering
    \includegraphics[ width = 1\columnwidth]{Figure/Appendix/6D/Fig_Scat_Agram-JND_Mean4_sad-ang.eps }
    \end{minipage}&
    
    \begin{minipage} {0.31\hsize}
    \centering
    \includegraphics [ width = 1\columnwidth]{Figure/Appendix/6D/Fig_Scat_Agram-JND_Mean4_hap-sad.eps }
    \end{minipage} &
    
    \begin{minipage} {0.31\hsize}
    \centering
    \includegraphics [ width = 1\columnwidth]{Figure/Appendix/6D/Fig_Scat_Agram-JND_Mean4_ang-hap.eps }
    \end{minipage} 

  \end{tabular}

  \vspace {-6pt}
  \caption{全実験参加者のJNDと平均聴力レベルとの相関分析結果。}

  \label{fig:CorrAud}

  \vspace {-12pt}
\end{figure}
%%%%%%%%%%%%%%%%%%%%%% %%%%%%%%%%%%%%%%%%%%% %%%%%%%%%%%%%%%%%%%%% %%%%%%%%%%%%%%%%%%%%% %%%%%%%%%%%%%%%%%%%%%% %%%%%%%%%%%%%%%%%%%%% %%%%%%%%%%%%%%%%%%%%% %%%%%%%%%%%%%%%%%%%%% 











































