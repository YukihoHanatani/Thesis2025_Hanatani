%%%%%%%%%%%%%%%%%%%%%%%%%%%%%%%%%%%%%%%%%%%%%%%%%%%%%%%%%%%%%%%%%%%%%%%%%%%%%%%%%%
%%%%%%%%%%%%%%%%%%%%%%%%%%%%%%%%%%%%%%%%%%%%%%%%%%%%%%%%%%%%%%%%%%%%%%%%%%%%%%%%%%
\chapter[GESI: Gammachirp Envelope Similarity Index]{GESI:\\ \fontsize{19pt}{19pt}\selectfont Gammachirp Envelope Similarity Index}
\label{chap:GESI}
%%%%%%%%%%%%%%%%%%%%%%%%%%%%%%%%%%%%%%%%%%%%%%%%%%%%%%%%%%%%%%%%%%%%%%%%%%%%%%%%%%
%%%%%%%%%%%%%%%%%%%%%%%%%%%%%%%%%%%%%%%%%%%%%%%%%%%%%%%%%%%%%%%%%%%%%%%%%%%%%%%%%%
節で述べたように音声強調処理の評価基準として広く利用されているSTOI\cite{taal2011algorithm}をはじめ、これまで数多くの音声了解度客観評価指標が提案されてきた。
しかし、これらの手法は健聴者に対応したもので、難聴者個人の聴覚特性を反映できない。
さらに、内部指標の算出方法を考慮すると、基準音と評価音に音圧レベル差がある場合や聴取環境による了解度変化の予測は困難である。
そこでこれらの課題を解決するため、新たな音声了解度客観評価指標Gammachirp envelope similarity index (GESI)を開発した。



%%%%%%%%%%%%%%%%%%%%%%%%%%%%%%%%%%%%%%%%%%%%%%%%%%%%%%%%%%%%%%%%%%%%%%%%
\section{GEDIの枠組みの活用}
%%%%%%%%%%%%%%%%%%%%%%%%%%%%%%%%%%%%%%%%%%%%%%%%%%%%%%%%%%%%%%%%%%%%%%%%
従来