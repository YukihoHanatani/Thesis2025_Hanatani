%
% $Id: thesis_sample.tex 809 2015-08-27 01:44:59Z fukuyasu $
%
\documentclass[11pt]{jreport}
\usepackage{wuse_thesis}
\usepackage{indentfirst}
\usepackage[dvipdfmx]{graphicx}  % ←graphicx.styを用いてEPSを取り込む場合有効にする


% 他のパッケージ・スタイルを使う場合には適宜追加
\usepackage{amsmath,amssymb,bm}  % 数式のパッケージ
\usepackage{subcaption}
\usepackage{enumerate} % 箇条書き
\usepackage{lscape} % 横向き
\usepackage{comment}
\usepackage{tipa} % IPS symbols essential for name of GESI/GEDI
\usepackage{cite}
\usepackage{otf}  % 宮﨑の漢字を入れるために必要  
                  % https://osksn2.hep.sci.osaka-u.ac.jp/~taku/osx/old/gaiji.html

% \usepackage[dvipdfmx]{color}
\usepackage{threeparttable}
\usepackage{here}
% \usepackage{url}	% \url{}コマンド用。URLを表示する際に便利

% Table
\usepackage{booktabs, multirow} % for borders and merged ranges
\usepackage{soul} % for underlines
\usepackage[table]{xcolor} % for cell colors
\usepackage{changepage,threeparttable} % for wide tables

\usepackage{otf} % ローマ数字
\usepackage{url} % 参考文献のURL

% \usepackage[dvipdfmx]{graphicx}

% \makeatletter
% \newcommand{\figcaption}[1]{\def\@captype{figure}\caption{#1}}
% \newcommand{\tblcaption}[1]{\def\@captype{table}\caption{#1}}
% \makeatother

% \usepackage{here}
% \usepackage{comment}
% \usepackage{caption}			
% \usepackage{latexsym}
% \usepackage{cite}
% \usepackage{pifont}

% \usepackage[dvipdfmx]{graphicx}
% \usepackage{subfig}
% \usepackage[dvipdfmx]{color}
% \usepackage{unicode-math}
% \usepackage[ipaex]{pxchfon}

% \usepackage{latexsym}
% \usepackage{amssymb}

% \usepackage{amssymb}
% \usepackage{amsmath}

% \usepackage[ipaex]{pxchfon}
% \usepackage{array, booktabs, multirow}
% \usepackage{wrapfig}
% \usepackage{subfig}
% \usepackage[dvipdfmx]{color}
% \newcommand{\argmax}{\mathop{\rm argmax}\limits}
% \newcommand{\argmin}{\mathop{\rm argmin}\limits}

% % Add, YA, January 2023
% \usepackage{tipa} %IPS symbols essential for name of GESI/GEDI
% \usepackage{lscape} % 横幅の大きい表を回転


%%%%%%%%%%%%%%%%%%%%%%%%%%%%%%%%%%%%%%%%%%%%%%%%%%%%%%%%%%%%%%%%%%%%%%%%

%%
%% 主に表紙を作成するための情報
%%

%%  タイトル(修論の場合は英語表記も指定)
\title{高齢難聴者の\\感情音声知覚特性についての研究(仮)}
\etitle{A Study of Emotional Speech Perception\\Characteristics of Older Persons with Hearing Loss()}

%%  著者名(修論の場合は英語表記も指定)
\author{花谷 幸歩}
\eauthor{Yukiho Hanatani}

%% 卒業論文・修士論文(以下のどちらかを選択)
%\bachelar	% 卒業論文(4年生用)
\master  	% 修士論文(M2用)

%%  学科・クラスタ
%\department{情報通信システム}
\department{システム知能}
%\department{デザイン科学}

%%  学生番号
\studentid{S2320107}

%%  卒業年度
\gyear{2025}		% 提出年が2015年なら,2014年度

%%  論文提出日
\date{2025年2月}	% 修士の場合は月(2015年2月)までとし,英語表記も指定
\edate{February 2025}	% 修士の場合,こちら(英語表記)も有効化

%%%%%%%%%%%%%%%%%%%%%%%%%%%%%%%%%%%%%%%%%%%%%%%%%%%%%%%%%%%%%%%%%%%%%%%%

\begin{document}

\maketitle

%%
%%  概要
%%


\begin{abstract}
超高齢社会を迎えた現在、高齢難聴者個人ごとに対応した音声・聴覚支援機材の開発が急務である。そこでは、音声強調処理/雑音抑圧処理が必須の技術となっている。この開発のためには、聴取者による処理音声の主観評価実験と、その結果を精度良く予測する客観評価指標が不可欠である。音声強調処理の評価基準として広く利用されているSTOIをはじめ、これまで数多くの客観評価指標が提案されてきた。しかし、これらの手法は健聴者に対応したもので、難聴者個人の特性を適切に反映できないものがほとんどである。

そこで本論文では、高齢難聴者の音声了解度予測を目指した新たな客観評価指標Gammachirp Envelope Similarity Index (GESI)を提案した。これは、基準音声と評価音声を、ガンマチャープ聴覚フィルタバンク(GCFB)と変調周波数フィルタバンク(MFB)の組み合わせで分析し、特徴量間の拡張コサイン類似度を計算して統合した指標である。
基準音声と評価音声の間の音圧レベル差や、聴取環境における閾値上レベルを適切に反映できる。

このGESIが3種類の主観評価実験結果を精度良く予測できるかを、従来手法と対比して評価した。
まず、模擬難聴音声システムWHISで高齢難聴者の聞こえにくさを模擬した音声に対する健聴者の主観評価結果を用いた。
WHISを用いたのは、実際の高齢難聴者実験で起こりうる機能低下の度合いや要因による結果のばらつきを抑え、聴覚末梢系の機能低下だけを評価するためである。これは、高齢難聴者個人ごとの予測の基礎になる。
従来手法STOIやその派生型は、内部指標計算時に行われる音圧正規化処理のため全く予測できなかった。
補聴器処理の評価のために提案されたHASPIは、聴取環境による了解度の変化を予測できなかった。
一方、GESIは話者性別や聴取環境(防音室/クラウドソーシングによる遠隔)にかかわらず、個人別に精度良く予測できた。
次に、模擬難聴処理に対して補聴器信号処理(処方式)を施した場合の主観評価了解度を予測した。GESIは、処方式による了解度の違いをHASPIよりも概ねよく予測できた。
これにより、補聴器装用時の音声了解度予測にも使用できる可能性が示唆された。
最後に、マルチチャンネルと理想的なシングルチャンネルの音声強調処理に対する健聴者の音声了解度を予測できるか検証した。
その結果、GESIは従来手法であるSTOIやその派生型と同程度の精度で予測できることがわかった。
これにより、難聴条件ばかりでなく、音声強調処理の開発全般にも利用できることが示された。

以上のことから、従来手法や補聴器信号処理の評価指標に取って代わる客観評価指標として有効であることが示された。

\end{abstract}

%%  目次
\tableofcontents

%%  図目次 (図目次をいれたければ以下のコメントをはずす)
%\listoffigures

%%  表目次 (表目次をいれたければ以下のコメントをはずす)
%\listoftables

\newpage
\pagenumbering{arabic}	% 以降のページ番号を算用数字に

%%%%%%%%%%%%%%%%%%%%%%%%%%%%%%%%%%%%%%%%%%%%%%%%%%%%%%%%%%%%%%%%%%%%%%%%

%%
%%  本文はここから
%%
%%%%%%%%%%%%%%%%%%%%%%%%%%%%%%%%%%%%%%%%%%%%%%%%%%%%%%%%%%%%%%%%%%%%%%%%


%%%%%%%%%%%%%%%%%%%%%%%%%%%%%%%%%%%%%%%%%%%%%%%%%%%%%%%%%%%%%%%%%%%%%%%%%%%%%%%%%%
%%%%%%%%%%%%%%%%%%%%%%%%%%%%%%%%%%%%%%%%%%%%%%%%%%%%%%%%%%%%%%%%%%%%%%%%%%%%%%%%%%
\chapter{はじめに}
\label{chap:Intro}
%%%%%%%%%%%%%%%%%%%%%%%%%%%%%%%%%%%%%%%%%%%%%%%%%%%%%%%%%%%%%%%%%%%%%%%%%%%%%%%%%%
%%%%%%%%%%%%%%%%%%%%%%%%%%%%%%%%%%%%%%%%%%%%%%%%%%%%%%%%%%%%%%%%%%%%%%%%%%%%%%%%%%


%%%%%%%%%%%%%%%%%%%%%%%%%%%%%%%%%%%%%%%%%%%%%%%%%%%%%%%%%%%%%%%%%%%%%%%%
\section{研究背景}
%%%%%%%%%%%%%%%%%%%%%%%%%%%%%%%%%%%%%%%%%%%%%%%%%%%%%%%%%%%%%%%%%%%%%%%%
\label{sec:研究背景}
超高齢社会となった日本では、加齢性難聴者の増加が懸念されている。
聴力低下は、音声による対人コミュニケーションを妨げ、生活の質(QOL)の低下を招く。
また、認知症リスクを高める最大要因であることがLancet委員会により報告されている\cite{livingston2020dementia}。
高齢者との円滑なコミュニケーションのためには、音声の言語情報ばかりではなく、感情を含む非言語情報も重要である。
特に感情(情動)は、言語情報と異なり曖昧で、高齢者にどの程度意図した通りに伝達できるかの特性もまだよくわかっていない。
これまでに、加齢により感情認識精度が低下することが報告されている\cite{paulmann2008aging,ben2019age,amorim2021changes}。
特に否定的な感情に関して低下するという報告もある\cite{mill2009age}。
また、従来からの補聴器で音声了解度の改善は行うことができるが、感情伝達特性の向上には貢献しないことが報告されている\cite{goy2018hearing}。
感情知覚機能の低下の要因は、聴覚末梢系の機能低下による聴力レベルによるものだけとは考えられてはいない。
しかしながら、明確にそのことを示す実験データは存在しないようである。
また、末梢系から認知系のどの部位の機能低下によるものかも解明されていない。
そこで、まずは聴覚末梢系の機能低下だけが、感情知覚にどの程度影響するかの調査を試みた。
若年健聴者に模擬難聴処理を行った音声を聴かせ、通常音声を聴いた場合との感情知覚の相違を実験的に検討した。
さらに、高齢者の感情知覚特性を、同じ通常音声を用いて同じ手順で測定した。
若年健聴者との相違から知覚特性の情報を得ることを目指した。

% SciRep
% 難聴(HL)を抱える高齢者の数は多くの国で増加している。
%  HLによる対人コミュニケーションの低下は、生活の質(QOL)を低下させるだけでなく、認知症の極めて高い危険因子であることが示されている1。
% 音声は、言語的内容と、話し手の特徴や感情を含む副言語的情報の両方を伝えます。
% 従来の補聴器は、主に音声の明瞭度を向上させるように設計されており、感情的なコミュニケーションを改善するものではない。
%  音声の感情認識2-5とその加齢効果6-14に関する研究は数多く行われており、加齢とともに感情認識・識別能力が低下することが報告されている。
% 補聴器は明瞭度を向上させるにもかかわらず、感情認識を向上させないことが確認されている9。
% 加齢は、蝸牛から中枢神経系に至る聴覚プロセスに影響を及ぼす。
% 一つの研究課題は、加齢に関連した末梢の聴覚障害が感情認知にどの程度影響するのか、また異なる段階でどのような過程が関与するのかということである。
% もう一つの疑問は、感情知覚をサポートする新しい補聴器を開発できるかどうかということである。
% 感情の低下が、認知のみによるのではなく、主に音声の特徴や声の表情の聴覚的表現の劣化によって引き起こされるのであれば、新しい強化アルゴリズムを開発することも可能であろう。

% 【copied 2022】世界に先駆けて超高齢化社会へ突入した日本では,今後も老人性難聴者が増加することは間違いない.
% 聴力低下による対人コミュニケーションの減少は,生活の質(Quality of life:QOL)を下げることに繋がる\cite{lancet}.
% また,老人性難聴は認知症の極めて高い要因であることが指摘されている\cite{AgeHL}.一般的に,難聴の対処には補聴器が挙げられる.
% その一方で,補聴器は難聴者の15\%ほどしか利用されておらず\cite{JPTrack},万人に有効な手立てとはなっていない.今までの補聴器は,
% 主に聞き取りやすさ(音声了解度)の改善だけを目的とされてきた.しかし,難聴者との円滑なコミュニケーションのためには音声了解度だけでなく,
% 感情の伝わりやすさ(感情伝達特性)についても考慮する必要があると考える.
% 老人性難聴者に話しかける際には「大きな声でゆっくりと話す」ことがよいとされている.ところが,これを表層的に捉えて単に大きな声で話しかけると,
% 「叱られている」と高齢者に勘違いされることがある.聴力レベルや言語情報処理から見た従来の理論においては正しいとされる方法であるにもかかわらず,
% 「感情」が伝わらないということは,感情伝達についての知見が不十分な証拠である.

% 本研究の将来的な目標は,老人性難聴者を含む他者への感情伝達特性を解明し,定式化することである.これにより,
% 補聴器の開発や超高齢社会における対人コミュニケーションの改善に繋がると考える.


%%%%%%%%%%%%%%%%%%%%%%%%%%%%%%
\section{研究目的}
%%%%%%%%%%%%%%%%%%%%%%%%%%%%%%
\label{sec:研究目的}

本研究の目的は、高齢者の感情知覚特性について調査することである。
将来的には、その特性を定式化し、補聴器の開発や高齢者との円滑なコミュニケーションに貢献することを目標としている。

本研究では、2つの実験を実施する。
1つ目は、若年健聴者実験である。
若年健聴者に難聴者の聞こえを体験させる「模擬難聴システムWHIS」(以下WHIS)を用いた聴取実験を行い、模擬難聴が感情知覚に与える影響を調査する。
通常音声を聞いた場合と、模擬難聴処理した音声を聞いた場合の結果に差異があるかどうかを確かめる。
WHISを用いて実験を行う理由は以下である。
加齢性難聴の原因は様々であり、聴覚末梢系の機能低下によるものか、中枢系・認知系の機能低下によるものかの切り分けが困難である。
どの原因がどの程度感情知覚に影響するのかを段階的に調査するために、まずは模擬難聴処理で聴覚末梢系の機能低下だけを模擬することにした。
したがって、WHISで処理した音声を聞いた若年健聴者を「模擬難聴者」として実験を行う(詳しくは第\ref{chap:ExpAngHapSad}章で説明する)。

2つ目は、高齢者実験である。
若年健聴者実験と同じ通常音声・手順で実験的に測定を行う。
若年健聴者結果と比較することで、聴覚末梢系の機能低下による感情知覚への影響があるかどうかを検証する。
また、聴覚末梢系の機能低下以外の要因も調査できると考え、実施することにした。

これらの実験から、若年健聴者が通常音声を聞いた場合・若年健聴者が模擬難聴処理音声を聞いた場合・高齢者が通常音声を聞いた場合の
感情音声の弁別精度を比較する。
以上の3条件の違いから高齢者の感情知覚特性について新たな知見を得ることを目的に実験を行った。


%SciRep
% 第一の研究課題に答えるためには、高齢者では変動が大きい聴覚経路や認知の要因を除外し、末梢HLのみがパーフォーマンスに及ぼす影響を特定する必要もある。
% この目的のために、健聴者(NH)にHL体験を提供するHLシミュレータを実験に使用することができる。
% HL模擬音を使った怒りの知覚に関する最近の実験がある14。
% 動機は似ているが、彼らの実験は基本的に感情識別課題であった。

%本研究では、音声モーフィングツールとHLシミュレータWHIS24を組み合わせて、感情ペア間の感情弁別実験を行った。
% 若年NH(YNH)参加者が通常の音を聞いた場合、同じYNH参加者がHL模擬音を聞いた場合、高齢参加者が同じ通常の音を聞いた場合の識別性能を比較した。
% これら3つの条件の違いから、高齢者の感情知覚の特徴について新たな知見が得られるかもしれない。

% 【copied 2022】本研究は,健聴者に難聴者の聞こえを体験させる「模擬難聴システムWHIS」(以下WHIS)を用いた聴取実験を行い,
% 模擬難聴が感情知覚に与える影響を調査することを目的とする.将来的な目標は,老人性難聴における感情伝達特性の解明と定式化である.


% WHISを用いて実験を行う理由は以下の2つである.第一に,実際の老人性難聴者を対象とした場合,難聴の原因が末梢系の機能低下によるものか,
% 認知機能の低下によるものかの切り分けが困難であり,個人ごとのばらつきも大きい.第二に,新型コロナウイルスの影響もあり高齢者を対象とした実験は難しい.
% このような理由から,WHISで処理した音声を聞いた健聴者を「模擬難聴者」として実験を行う(詳しくは,第\ref{sec:感情知覚実験}章で説明する).

% 本研究では,同一被験者で健聴状態と模擬難聴状態で聴取実験を行い,感情判断の差異をもとに模擬難聴が感情知覚に及ぼす影響について調査する.


%%%%%%%%%%%%%
\section{本論文の構成}
%%%%%%%%%%%%%
\label{sec:本論文の構成}
本論文は、本章を含めて5章で構成されている。
第1章では、本研究の背景と目的について示した。
第2章では、聴覚に関する基本的な知識と、音声と感情知覚に関する従来研究について説明する。
第3章では、怒り・悲しみ・喜び間の弁別実験について、詳細な手続きを述べた上で、結果についてまとめる。
第4章では、落着きと怒り・悲しみ・喜び間の弁別実験について、手続きと結果についてまとめる。
また、第3章の怒り・悲しみ・喜び実験の結果と比較検討する。
最後に、第5章で本研究について総括する。

% 【copied 2022】本論文は,本章を含め6章で構成されている.第1章では,本研究の背景と目的について示した.第2章では,聴覚に関する基本的な知識と,音声と感情知覚に
% 関する知識,先行研究について説明する.第3章では,本実験の詳細な手続きを述べる.第4章では実験結果と統計的分析を行なった結果を示し,第5章で考察を述べる.
% 最後に,第6章にて本研究についてまとめる.


% \include{2_GESI}
% \include{3_WHIS} % 他と違うところをいう
% \include{4_HA}
% \include{5_MulCh} % ベースと同じくらいできる
% %%%%%%%%%%%%%%%%%%%%%%%%%%%%%%%%%%%%%%%%%%%%%%%%%%%%%%%%%%%%%%%%%%%%%%%%%%%%%%%%%%
%%%%%%%%%%%%%%%%%%%%%%%%%%%%%%%%%%%%%%%%%%%%%%%%%%%%%%%%%%%%%%%%%%%%%%%%%%%%%%%%%%
\chapter{総括}
%%%%%%%%%%%%%%%%%%%%%%%%%%%%%%%%%%%%%%%%%%%%%%%%%%%%%%%%%%%%%%%%%%%%%%%%%%%%%%%%%%
%%%%%%%%%%%%%%%%%%%%%%%%%%%%%%%%%%%%%%%%%%%%%%%%%%%%%%%%%%%%%%%%%%%%%%%%%%%%%%%%%%
%%%%%%%%%%%%%%%%%%%%%%%%%%%%%%%%%%
\section{本論文のまとめ}
\label{sec:}
%%%%%%%%%%%%%%%%%%%%%%%%%%%%%%%%%%
本報

%%%%%%%%%%%%%%%%%%%%%%%%%%%%%%%%%%%%%%%%%%%%%%%%%%%%%%%%%%%%%%%%%%%%%%%%
%謝辞
\begin{acknowledgements}
本研究を進めるにあたり、本学システム工学研究科の入野俊夫教授には指導教官として終始あたたかいご指導を賜りました。厚く御礼申し上げます。
% 本研究を行うにあたり、豊富な知識と経験の下、音声強調処理に関する知見のご教授や多大なご助言を賜りました日本電信電話株式会社 コミュニケーション科学基礎研究所の荒木章子氏、新井賢一氏、小川厚徳氏、木下慶介氏、中谷智広氏に心より感謝申し上げます。
% 本研究のデータ収集・聴取実験に協力していただいた和歌山大学システム工学部 宮\UTF{FA11}芙紀氏、同 卒業生の田丸萌夏氏に深く感謝いたします。本研究の実験参加者として、ご理解とご協力を賜った本学およびLancers利用者の皆様に心より御礼申し上げます。
% 本研究を遂行するにあたり、語彙数推定テストのご教授とリストを作成をしてくださった、愛知淑徳大学 天野成昭教授に感謝いたします。
% 国内会議や国際会議で本研究に興味を持っていただき、多くの議論や交流をしていただいた皆様に深く感謝申し上げます。

% インターンシップ期間中、研究に向かう姿勢や主観評価実験手法の知見をご教示賜りました日本電信電話株式会社 コミュニケーション科学基礎研究所 厚木研究開発センターの皆様、ソニーグループ株式会社 R\&Dセンター Tokyo Laboratoryの皆様に心よりお礼申し上げます。また期間中、切磋琢磨しながら共に有意義な時間を過ごしてくださったインターンシップ生の皆様に感謝いたします。
% 多種多様な事務手続きなど多大なサポートをしていただいた和歌山大学職員の谷本英美子氏に感謝申し上げます。
% 聴覚メディア研究室の諸氏には、日頃から研究・講義・その他広範囲に渡り支えていただきました。ここに深く感謝致します。
\end{acknowledgements}

%%%%%%%%%%%%%%%%%%%%%%%%%%%%%%%%%%%%%%%%%%%%%%%%%%%%%%%%%%%%%%%%%%%%%%%%

%%
%% 参考文献
%%

\bibliographystyle{junsrt}
% \bibliography{Reference_Oct24}
\bibliography{Reference_29Dec24}


%%%%%%%%%%%%%%%%%%%%%%%%%%%%%%%%%%%%%%%%%%%%%%%%%%%%%%%%%%%%%%%%%%%%%%%%

%%
%% 付録
%%
\appendix
% \include{7_Appendix}
% \include{7_Appendix_FigSbj_OIMHL}
% \include{7_Appendix_FigSbj_fOIMHL}
% \include{7_Appendix_FigSbj_OIMHL_HASPI}
% \include{7_Appendix_FigSbj_fOIMHL_HASPI}

\newpage
%%%%%%%%%%%%%%%%%%%%%%%%%%%%%%%%%%%%%%%%%%%%%%%%%%%%%%%%%%%%%%%%%%%%%%%%
\section*{業績一覧}
%%%%%%%%%%%%%%%%%%%%%%%%%%%%%%%%%%%%%%%%%%%%%%%%%%%%%%%%%%%%%%%%%%%%%%%% 
%%%%%%%%%%%%%%%%%%%%%%%%%%%%%%%%%%%%%%%%%%%%%%%%%%%%%%%%%%%%%%%%%%%%%%%%
\subsection*{国際会議(査読あり)}
%%%%%%%%%%%%%%%%%%%%%%%%%%%%%%%%%%%%%%%%%%%%%%%%%%%%%%%%%%%%%%%%%%%%%%%%
 \begin{enumerate}

    \item \textbf{Ayako Yamamoto}, Toshio Irino, Kenichi Arai, Shoko Araki, Atsunori Ogawa, Keisuke Kinoshita, Tomohiro Nakatani, "Comparison of Remote Experiments Using Crowdsourcing and Laboratory Experiments on Speech Intelligibility," Proc. Interspeech 2021, pp.181--185, Brno, Czech Republic \& online, 30 August -- 3 September 2021.
    % DOI: 10.21437/Interspeech.2021-174.

    % \item Toshio Irino, Honoka Tamaru, \textbf{Ayako Yamamoto}, "Speech intelligibility of simulated hearing loss sounds and its prediction using the Gammachirp Envelope Similarity Index (GESI) ," Proc. Interspeech 2022, pp.3929--3933, Incheon, Korea, 18--22 September 2022.
    % % DOI: 10.21437/Interspeech.2022-211.

    % \item \textbf{Ayako Yamamoto}, Toshio Irino, Shoko Araki, Kenichi Ara, Atsunori Ogawa, Keisuke Kinoshita, and Tomohiro Nakatani "Effective data screening technique for crowdsourced speech intelligibility experiments: Evaluation with IRM-based speech enhancement," Proc. APSIPA ASC 2022, pp.1402--1408, Chiang Mai, Thailand, 7--10 November, 2022.
\end{enumerate}

%%%%%%%%%%%%%%%%%%%%%%%%%%%%%%%%%%%%%%%%%%%%%%%%%%%%%%%%%%%%%%%%%%%%%%%%
\subsection*{国内発表}
%%%%%%%%%%%%%%%%%%%%%%%%%%%%%%%%%%%%%%%%%%%%%%%%%%%%%%%%%%%%%%%%%%%%%%%%
 \begin{enumerate}
    \item \textbf{山本絢子}, 入野俊夫, 新井賢一, 荒木章子, 小川厚徳, 木下慶介, 中谷智広, "クラウドソーシングを利用した音声了解度実験 --- ウェブページ制作からデータスクリーニング ---," 音学シンポジウム2021, 発表番号33, 電子情報通信学会, 音声研究会, 電子情報通信学会技術研究報告, SP2021-5, pp.25--30, オンライン, 2021年6月18日--19日.
    
    % \item \textbf{山本絢子}, 入野俊夫, 新井賢一, 荒木章子, 小川厚徳, 木下慶介, 中谷智広, "マルチチャンネル音声強調処理の主観評価," 日本音響学会関西支部, 第24回関西支部若手研究者交流研究発表会, 発表番号43, オンライン, 2021年12月4日.

    % \item \textbf{山本絢子}, 入野俊夫, 新井賢一, 荒木章子, 小川厚徳, 木下慶介, 中谷智広, "IRMを用いた音声強調処理の主観了解度の上限評価 --- 防音室実験とクラウドソーシング実験の対比 ---," 音声研究会, 電子情報通信学会技術研究報告, SP2021-59, EA2021-74, SIP2021-101, p.64--69, 沖縄県立博物館/オンライン, 2022年3月1日--2日.

    % \item \textbf{山本絢子}, 入野俊夫, 新井賢一, 荒木章子, 小川厚徳, 木下慶介, 中谷智広, "MVDRビームフォーマーによる音声強調処理の了解度評価 --- 防音室実験とクラウドソーシング実験の対比 ---," 日本音響学会春季研究発表会講演論文集, 1-1P-8, pp.323--326, オンライン, 2022年3月9日--11日.

    % \item 入野俊夫, 田丸萌夏, \textbf{山本絢子}, "模擬難聴システムWHISの新実装と末梢系特性の音声了解度への影響," 日本音響学会春季研究発表会講演論文集, 2-4-13, pp.665--668, オンライン, 2022年3月9日--11日.

    % \item 上野朱音, 入野俊夫, \textbf{山本絢子}, "異なる身長の小学生の音声を用いた寸法知覚実験," 日本音響学会春季研究発表会講演論文集, 3-4Q-6, pp.767--768, オンライン, 2022年3月9日--11日.

    % \item 入野俊夫, 田丸萌夏, \textbf{山本絢子}, "Gammachirp Envelope Similarity Index (GESI) による模擬難聴音声の了解度予測 ~ 防音室実験とクラウドソーシング遠隔実験の主観評価データを用いて ~", 音学シンポジウム2022, 発表番号49, 電子情報通信学会, 音声研究会, 電子情報通信学会技術研究報告, SP2022-17, pp.71--76, オンライン開催, 2022年6月17日--18日.

    % \item \textbf{山本絢子}, 入野俊夫, 荒木章子, 田丸萌夏, 新井賢一, 小川厚徳, 木下慶介, 中谷智広, "客観評価指標GESIによる音声了解度予測 --- 強調処理音声と音圧低減音声を対象として ---, "日本音響学会聴覚研究会資料, Vol.52, No.5, H-2022-65, 音声研究会, SP2022-41, 電子情報通信学会技術研究報告, EA2022-25, pp.345--350, 北海道大学, 2022年7月7日--8日.
    
    % \item \textbf{山本絢子}, 入野俊夫, 荒木章子, 田丸萌夏, 新井賢一, 小川厚徳, 木下慶介, 中谷智広, "高齢難聴者の音声了解度客観評価を目指したGESIの開発 --- 強調音声と模擬難聴音声による評価 ---," 日本音響学会秋季研究発表会講演論文集, 3-P-27, pp.981--984, 北海道科学大学, 2022年9月14日--16日.

    % \item \textbf{山本絢子}, 宮\UTF{FA11}芙紀, 田丸萌夏, 入野俊夫, "模擬難聴音声了解度の主観評価実験とGESIによる予測," 日本音響学会関西支部, 第25回関西支部若手研究者交流研究発表会, 発表番号21, 同志社大学 京田辺キャンパス, 2022年11月26日.
    
    % \item 宮\UTF{FA11}芙紀, \textbf{山本絢子}, 土庵晋太郎, 入野俊夫, "クラウドソーシング聴取実験のための効果的な事前参加者スクリーニングの検討," 日本音響学会関西支部, 第25回関西支部若手研究者交流研究発表会, 発表番号48, 同志社大学 京田辺キャンパス, 2022年11月26日.

    % \item \textbf{山本絢子}, 宮\UTF{FA11}芙紀, 田丸萌夏, 入野俊夫, "客観評価指標 GESI による模擬難聴音声の了解度予測 --- 健聴者による原音声の主観評価値のみを用いて ---," 日本音響学会聴覚研究会資料, Vol.52, No.8, pp.641--646, H-2022-117, 九州大学大橋キャンパス, 2022年12月17日--18日.

    % \item \textbf{山本絢子}, 宮\UTF{FA11}芙紀, 田丸萌夏, 入野俊夫, "客観評価指標GESIによる模擬難聴音声了解度の個人別予測," 日本音響学会春季研究発表会講演論文集, 3-4P-7, オンライン, 2023年3月15日--17日.
    
    % \item 宮\UTF{FA11}芙紀, \textbf{山本絢子}, 土庵晋太郎, 入野俊夫, "クラウドソーシング聴取実験のための効果的な事前参加者スクリーニング," 日本音響学会春季研究発表会講演論文集, 3-4P-8, オンライン, 2023年3月15日--17日.
\end{enumerate}

\end{document}