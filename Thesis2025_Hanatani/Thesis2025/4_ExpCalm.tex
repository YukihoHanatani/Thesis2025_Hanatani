\newpage
%%%%%%%%%%%%%%%%%%%%%%%%%%%%%%%%%%%%%%%%%%%%%%%%%%%%%%%%%%%%%%%%%%%%%%%%%%%%%%%%%%
%%%%%%%%%%%%%%%%%%%%%%%%%%%%%%%%%%%%%%%%%%%%%%%%%%%%%%%%%%%%%%%%%%%%%%%%%%%%%%%%%%
\chapter{落着きと怒り・悲しみ・喜び間の弁別実験}
\label{chap:ExpCalm}
%%%%%%%%%%%%%%%%%%%%%%%%%%%%%%%%%%%%%%%%%%%%%%%%%%%%%%%%%%%%%%%%%%%%%%%%%%%%%%%%%%
%%%%%%%%%%%%%%%%%%%%%%%%%%%%%%%%%%%%%%%%%%%%%%%%%%%%%%%%%%%%%%%%%%%%%%%%%%%%%%%%%%
第\ref{chap:ExpAngHapSad}章の実験では、高齢者にとって喜び--怒りの判断だけが有意に難しいことがわかった。
そこで、ラッセルの感情円環モデル\cite{russell1980circumplex}との対応関係を用いて考察を進めた。
そして、覚醒度が同程度である場合、快-不快の判断が難しいという作業仮説を立てた。
その上で、ラッセルの円環モデル上で悲しみと快-不快軸上で反対にある「落着き」の音声を使って同様な弁別実験を行った。

ここでは、第\ref{chap:ExpAngHapSad}章と同様の構成で、実験手順・実験条件・実験結果について述べる。
実験結果では、第\ref{chap:ExpAngHapSad}章で行った怒り・悲しみ・喜び間の弁別実験結果と比較することで、作業仮説の検証を行う。

% ------------------------------
\section{実験刺激の作成}
\label{sec:PrepareStimuli_cal}
% ------------------------------
怒り・悲しみ・喜び実験(第\ref{chap:ExpAngHapSad}章)と同様な手順で、「落着き」感情を最もよく表現する音声の抽出を行い、
音声モーフィングと模擬難聴処理で実験刺激を作成した。


% ------------------------------
\subsubsection{「落着き」感情音声の選別}
% ------------------------------
使用した元音声は、怒り・悲しみ・喜び実験と同じ慶應義塾大学研究用感情音声データベース(Keio-ESD)\cite{keioESD-J}である。
用いた単語も同じで、名詞の10単語とした。

今回の実験では、上述の作業仮説を検証するため、「快」かつ「沈静」の感情である「落着き」を新たに導入した。
実験者3名(著者を含む若年健聴者:日本人大学生と大学院生)で47元感情・10単語の音声を再度全て聞き、「落着き」に感じた発声(48音声)を選別した。
この48音声に、怒り・悲しみ・喜び実験で用いた30音声を加えた合計78音声を用いて、次の感情尺度評定を行っていく。
詳細な手続きは付録\ref{sec:DBScreening}にて述べる。

% \newpage


% ------------------------------
\subsubsection{感情尺度評定と主成分分析}
\label{sec:5Scale_Calm}
% ------------------------------
選別した78音声に関して、実験者3名が「怒り」「悲しみ」「喜び」「落着き」の4感情を評定軸とした段階尺度評定を行った。
その上で、全音声・実験者3名の4感情の評定値を変数として主成分分析を行った。
この結果の第3主成分(PC3)までの分布を図\ref{fig:PCA-Russel_Calm}(c)に示す。
PC3までで、累積寄与率99\%を占めた。
図\ref{fig:PCA-Russel_AngHapSad}(a)と同様に、「怒り」「悲しみ」「喜び」の3感情が三角形の頂点付近にそれぞれ分布している。
一方、落着き音声(黒のx印)は3感情とは逆方向に、三角形を底面とした三角錐の頂点の方向に分布している。
各単語ごとに、落着きベクトル(Cal)との内積が最も大きい、つまり最も「落着き」らしい10単語を抽出した。
以上より、同一内容の10単語を4感情分そろえた。


%%%%%%%%%%%%%%%%%%%%%% %%%%%%%%%%%%%%%%%%%%% %%%%%%%%%%%%%%%%%%%%% 
% 主成分分析結果・Russel  Calm

% ---------------------------------------
\begin{figure}[h]
  \vspace{40pt}
  \hspace{20pt}
  \centering
  \includegraphics[width=0.8\hsize]{Figure/ExpCalm/FigPCA3d_CumAll_cal2_scatter_SndTriangle_Eng.eps}
  \caption{感情尺度評定の主成分分析(PCA)結果。各点(x)は、怒り,悲しみ,喜び,落着きの評価を行って得られたPCAスコアを示す。
            "Ang"怒り,"Sad"悲しみ,"Hap"喜び,"Cal"落着き。 見やすさのため3次元で描画している。}
\end{figure}
% --------------------------------------------


% \begin{figure}[t]

%   \begin{tabular}{cc}
%   \begin{minipage} {0.47\hsize}
%   \centering
%   \includegraphics [ width = 1\columnwidth]{Figure/ExpAngHapSad/FigPCA_CumAll_scatter_SndTriangle_Eng.eps}
%   \end{minipage} & 
  
%   \begin{minipage} {0.47\hsize}
%   \centering
%   \includegraphics [ width = 1\columnwidth]{Figure/ExpAngHapSad/Fig_RusselCircle_b.eps }
%   \end{minipage}
  
%   \end{tabular}
  
%   \caption{感情尺度評定の主成分分析(PCA)結果(a)とラッセルの感情円環モデル(b)(\cite{russell1980circumplex}のFig.4より再描画)。
%             図(a)の各点(x)は、単語ごとに基本6感情("Ang"怒り,"Sad"悲しみ,"Hap"喜び,"Fea"恐怖,"Dis"嫌悪,"Sur"驚き)の評価を行って得られたPCAスコアを示す。
%             薄い線の三角形は抽出された10単語それぞれを結ぶ。
%             ラッセルの円環モデル(b)内の感情との位置関係に合わせるため、横軸をPC2、縦軸をPC1の符号反転としている。
%             }
%   \label{fig:PCA-Russel_Calm} 

% \end{figure}




% \newpage
% ------------------------------
\subsubsection{音声モーフィング}
\label{sec:morphCalm}
% ------------------------------

抽出した40音声(10単語$\times$4感情)について、音声モーフィング用のGUI\cite{kawahara2024interactive}を用いて、
「怒り--落着き」「悲しみ--落着き」「喜び--落着き」間の中間の音声を合成した。
合成の手順は怒り・悲しみ・喜び実験と同じで、モーフィング率が0,20,40,50,60,80,100\%の音声を、3つの感情対に関して10単語分作成した。 
図\ref{fig:ExpRsltEmoPercent_Calm}(a)にその音声の配置の概念図を示す。
これにより、10単語$\times$7モーフィング率$\times$3感情対の合計210音声を準備した。

% ------------------------------
\subsubsection{模擬難聴処理}
% ------------------------------

怒り・悲しみ・喜び実験と同様に、若年健聴者実験では通常音声と模擬難聴音声の対比を行う。
模擬難聴システムWHIS\cite{irino2023hearing}を用いて、模擬難聴音声を作成した。
80歳の平均聴力レベル \cite{tsuiki2002nihon_Jpn}、圧縮特性健全度を中程度の$\alpha=0.5$と設定した。


\newpage
% ------------------------------
\section{実験手順と実験条件}
\label{sec:ExpCondition_Calm}
% ------------------------------
前節で述べた実験刺激を用いて、若年健聴者と高齢者を対象として実験を行った。
実験手順は怒り・喜び・悲しみ実験と同じである。
参加者は「怒り--落着き」対、「悲しみ--落着き」対、「喜び--落着き」対の3つの実験すべてに参加した。
実験は和歌山大学の倫理委員会の承認を受けており、実験前に説明を行ってインフォームドコンセントを得た。


% ------------------------------
\subsubsection{若年健聴者実験}
% ------------------------------

若年健聴者実験では、無処理音(Unpro)と模擬難聴音(80yr)の両方を使った。
したがって、刺激対は、10単語$\times$標準刺激と比較刺激の組み合わせ6組$\times$音声処理条件(Unpro/80yr)2条件$\times$提示順のカウンタバランス2条件=240対である。
提示順は基本的に参加者ごとにランダムで、1セッションあたり各12の音声対の全20セッションとした。
Unproと80yrは音圧が異なるため、別セッションに割り当てて、セッション内ではどちらかの条件に統一されている。
また、感情判断においてバイアスがないかを調べるために、判断させる感情を入れ替えて2回実験した(詳細は\ref{sec:ExpCondition}節を参照)。
すなわち、1人の実験参加者は、全480対を聴取した。

参加者は、21歳から25歳の日本人大学生12名(男女6名ずつ)であった。
全員が125Hzから8000Hzの範囲で聴力レベルが20\,dB以下で健聴者であることを確認した。


% ------------------------------
\subsubsection{高齢者実験}
% ------------------------------
高齢者実験は若年健聴者実験と同様の手順である。
ただし、Unpro音声だけを実験刺激として、80yrの模擬難聴音ははずした。
したがって刺激対は、10単語$\times$標準刺激と比較刺激の組み合わせ6組$\times$提示順のカウンタバランス2条件=120対である。
健聴者実験の半分なので全10セッションとした。
感情判断に用いる手掛語は若年健聴者実験と同様に2通り設定した。
すなわち、1人の実験参加者は、全240対を聴取した。
どちらの手掛語から先に始めるかは、全被験者で統一した。
どの感情対の実験から始めるかは、参加者間でバランスを取った。

シルバー人材センターに依頼して実験参加者を募集した。
参加者は62歳から82歳の12名(男9名・女3名)で、内10名は怒り・喜び・悲しみ実験にも参加している。
%怒り・喜び・悲しみ実験には12名参加していたが、内2名がシルバー人材センターを退会したため、今回は新たに2名追加した。 --- 書く必要なし
参加者のうち3名は、良耳が125Hzから8000Hzの全帯域で健聴レベルであった。
しかし、残りは高齢難聴の傾向があった。
500~Hzから4000~Hzの平均聴力レベルは3.8$\sim$51.3dBであった。


%%%%%%%%%%%%%%%%%%%%%%%%%%%%%%%%%
\subsubsection{実験機材と提示条件}
%%%%%%%%%%%%%%%%%%%%%%%%%%%%%%%%%%
若年健聴者と高齢者で同じ条件で実験を行った。
聴取実験を暗騒音レベルが約26\,dBの防音室(YAMAHA AVITECS)内で実施した。
実験に使用する音響機材は、怒り・喜び・悲しみ実験と同様である(詳細は\ref{sec:ExpCondition}節を参照)。
若年健聴者・高齢者ともに、Unproの音声の提示音圧レベルが${L_{eq}}$ で65\,dBとなるようにした。
聴力レベルが低下していても、十分に聞こえることは確認した。

\newpage
% ------------------------------
\section{実験結果}
\label{sec:ResultCalm}
% ------------------------------
怒り--落着き対、悲しみ--落着き対、喜び--落着き対に関して、標準刺激(50\%モーフィング音)に対して、
比較刺激の方が手掛語に示された感情に近いと感じた回答率を実験参加者ごとに算出した。
結果を図\ref{fig:ExpRsltEmoPercent_Calm}(b)$\sim$(g)に示す。
図\ref{fig:ExpRsltEmoPercent_Calm}(a)は、\ref{sec:morphCalm}節で述べたモーフィング音声の配置の概念図である。


%%%%%%%%%%%%%%%%%%%%%%%%%%%%%%%%%
\subsection{若年健聴者実験結果}
%%%%%%%%%%%%%%%%%%%%%%%%%%%%%%%%%%
図\ref{fig:ExpRsltEmoPercent_Calm}中段に、(b)怒り-落着き対、(c)悲しみ-落着き対、(d)喜び-落着き対の回答率の、参加者間の平均値と標準偏差を示す。
なお、\ref{sec:ExpCondition_Calm}節で述べたように感情判断の手掛語は一つの感情ともう片一方の感情を使った。
感情A-感情Bの対の場合、より感情Aと感じる方を判断させる場合(回答率A\%)と、より感情Bと感じる方を判断させる場合(回答率B\%)である。
2本の回答曲線を比較しやすいように、Bの回答率は100\%から引いて(100-B)\%で表示している。
図には、2つの手掛語、模擬難聴処理の有無(Unpro/80yr)の合計4本の回答曲線を表示している。

結果的に、図\ref{fig:ExpRsltEmoPercent_Calm}(b),(c),(d)のいずれの対においても、回答率A\%と(100-B)\%の回答曲線はほとんど重なっていた。
このことから、手掛語は若年健聴者の感情知覚にほとんど影響しないことがわかった。
また、全ての対でUnproと80yrの回答曲線がほとんど重なっており、模擬難聴処理は感情弁別に影響していないことを意味する。
すなわち、模擬している聴覚末梢系の機能低下だけは、感情知覚に影響しないことを示唆する。
これらの結果は、怒り・悲しみ・喜び実験と同様である。

さらに、感情対間を比較しても、回答曲線における判断が50\%になる点はほとんど差異がない。
傾きに関しては若干違いがあるように見えるため、\ref{sec:Statistics_Calm}項で統計的に分析する。

%%%%%%%%%%%%%%%%%%%%%%%%%%%%%%%%%
\subsection{高齢者実験結果}
%%%%%%%%%%%%%%%%%%%%%%%%%%%%%%%%%%
図\ref{fig:ExpRsltEmoPercent_Calm}下段に、(e)怒り--落着き対、(f)悲しみ--落着き対、(g)喜び--落着き対の実験結果を示す。
若年健聴者と同様に、図\ref{fig:ExpRsltEmoPercent_Calm}(e),(f),(g)のいずれの対においても、回答率A\%と(100-B)\%はほとんど重なっており、手掛語の影響はないことがわかった。

図\ref{fig:ExpRsltEmoPercent_Calm}(b),(c),(d)の若年健聴者結果と比較すると、全体的に傾きが緩やかになっている。
3つの感情対の中では、図\ref{fig:ExpRsltEmoPercent_Calm}(f)の悲しみ--落着き対が最も傾きが急であることがわかる。
個人ごとに回答率を見ると、怒り--落着き、喜び--落着き対ではかなり緩やかな傾きになる参加者が多くいた一方、悲しみ--落着き対では若年健聴者と同程度の精度で弁別できた参加者が見られた。
また、多くの参加者がどの感情対も難しかったと内観報告していたが、悲しみ--落着き対が最も難しかったと回答した参加者はいなかった。







%%%%%%%%%%%%%%%%%%%%% %%%%%%%%%%%%%%%%%%%%% %%%%%%%%%%%%%%%%%%%%% 
% モーフィングの概念図と心理物理曲線(Calm)
%%%%%%%%%%%%%%%%%%%%%% %%%%%%%%%%%%%%%%%%%%% %%%%%%%%%%%%%%%%%%%%% %%%%%%%%%%%%%%%%%%%%% 
\begin{figure}[t]
  % \vspace {-20pt}
  %%%%%%%%%%%%%%%%%%%%% 上段 morphing 図 %%%%%%%%%%%%%%%%%%%%% 
  
  \begin{center}
  
  %\includegraphics [ width = 0.35\columnwidth]{FigPCA_morphing.png}
  \includegraphics [ width = 0.65\columnwidth]{Figure/ExpCalm/FigPCA_morphRatioArrow_Calm_Eng.eps}
  \end{center}
  %\caption{Conceptual diagram of placement of emotional morphing voices. Morphing ratio: ×: 50\%, o: 20,40,60,80\% }
  % \label{fig:ExpEmoWHIS _ang-hap-sad }
  
  % \end{figure}
  %----------------------------------%
  
  % \begin{figure}[t]
  \vspace {-12pt}
  \begin{tabular}{ccc}
  %%%%%%%%%%%%%%%%%%%%% 中段 %%%%%%%%%%%%%%%%%%%%% 
  
  \begin{minipage} {0.31\hsize}
  \centering
  \includegraphics[ width = 1\columnwidth]{Figure/ExpCalm/FigYNH_Raw_AllSbj_cal-ang.eps }
  \end{minipage}&
  
  \begin{minipage} {0.31\hsize}
  \centering
  \includegraphics [ width = 1\columnwidth]{Figure/ExpCalm/FigYNH_Raw_AllSbj_cal-sad.eps }
  
  \end{minipage} &
  
  \begin{minipage} {0.31\hsize}
  \centering
  \includegraphics [ width = 1\columnwidth]{Figure/ExpCalm/FigYNH_Raw_AllSbj_cal-hap.eps }
  
  \end{minipage} 
  
      
  \\  %% 改行  %%%%%%%%%%%%%%%%%%%%% 
  
  %%%%%%%%%%%%%%%%%%%%% 下段 %%%%%%%%%%%%%%%%%%%%% 
  
  
  \begin{minipage} {0.31\hsize}
  \centering
  \includegraphics [ width = 1\columnwidth]{Figure/ExpCalm/FigEld_Raw_AllSbj_cal-ang.eps }
  \end{minipage}&
  
  \begin{minipage} {0.31\hsize}
  \centering
  \includegraphics [ width = 1\columnwidth]{Figure/ExpCalm/FigEld_Raw_AllSbj_cal-sad.eps }
  \end{minipage} &
  
  \begin{minipage} {0.31\hsize}
  \centering
  \includegraphics [ width = 1\columnwidth]{Figure/ExpCalm/FigEld_Raw_AllSbj_cal-hap.eps }
  \end{minipage}
  
  \end{tabular}
  
  \vspace {-6pt}
  % \caption{ Results of the emotion discrimination experiments. The top panel (a) shows a schematic plot of the stimulus sounds with morphing ratios of 50\% (x) and 20\%, 40\%, 60\%, and 80\% (o) between the emotions ``anger'' (Ang), ``sadness'' (Sad), and ``happiness'' (Hap), respectively. The morphing ratio of Ang relative to Hap is shown as an example.
  % The middle panels show the means and standard deviations of the percentage responses across YNH participants for the Ang-Sad pair (b), the Sad-Hap pair (c), and the Hap-Ang pair (d). The bottom panels show those across older participants for the Ang-Sad pair (e), the Sad-Hap pair (f), and the Hap-Ang pair (g). Horizontal axis: Vocal morphing ratio (\%).  Vertical axis: Percent response (\%) of Sad or 100-Ang, Hap or 100-Sad, and Ang or 100-Hap. Line colors correspond to cue words in emotion judgments.
  % }
  \caption{感情弁別実験の刺激音配置と結果。
            上図(a)は、感情「怒り」(Ang)、「悲しみ」(Sad)、「喜び」(Hap)、「落着き」(Cal)のモーフィング率が50\%(x)、20\%、40\%、60\%、80\%(o)の刺激音配置の模式図。
            中段は若年健聴者全体の回答の割合の平均と標準偏差を示していて、ペアが怒-落(b)、悲-落(c)、喜-落(d)の場合である。
            下図は高齢者全体の回答で、怒-落ペア(e)、悲-落ペア(f)、喜-落ペア(g)の場合である。
            横軸: モーフィング率(\%)。 縦軸: Calまたは100-Ang、Calまたは100-Sad、Calまたは100-Hapの回答率。}
  
  \label{fig:ExpRsltEmoPercent_Calm}

  \vspace {-12pt}
  \end{figure}
  %%%%%%%%%%%%%%%%%%%%%% %%%%%%%%%%%%%%%%%%%%% %%%%%%%%%%%%%%%%%%%%% %%%%%%%%%%%%%%%%%%%%% 





\clearpage
%%%%%%%%%%%%%%%%%%%%%%%%%%%%%%%%%
\subsection{統計的分析}
\label{sec:Statistics_Calm}
%%%%%%%%%%%%%%%%%%%%%%%%%%%%%%%%%%
上記の観察結果を統計的いに明確化するため、図\ref{fig:ExpRsltEmoPercent_Calm}(b)$\sim$(g)の回答曲線に累積ガウス分布を当てはめ、心理物理曲線を推定した。
そして、主観的等値点(PSE)と弁別閾(JND)を算出した。
図\ref{fig:JNDPSE_Calm}(a)にJND、(b)にPSEの、参加者間の平均値と95\%信頼区間を示す。
上述したように、手掛語による影響はほとんどみられなかったため、反復測定として感情対ごとにまとめた。
図左側が若年健聴者模擬難聴なし(YNH-Unpro)、中央が若年健聴者模擬難聴あり(YNH-80yr)、右側が高齢者(Older-Unpro)の結果である。
また、9条件(YNH-Unpro・YNH-80yr・Older-Unpro、感情対すべて)に関してTukey HSDの多重比較検定(有意水準5\%)を行った。

%%%%%%%%%%%%%%%%%%%%%%%%%%%%%%%%%
\subsubsection{(a)JND}
%%%%%%%%%%%%%%%%%%%%%%%%%%%%%%%%%%
図\ref{fig:JNDPSE_Calm}(a)のJNDの平均値に関して、若年健聴者で模擬難聴処理の有無で傾向に差異はなかった。
怒り--落着き対の値が他の感情対より若干大きくなっているが、有意差はなかった。
一方、高齢者においては全感情対で若年健聴者よりも10\%ほど大きな値となった。
多重比較検定の結果、高齢者の怒り--落着き(A-C)対・喜び--落着き(H-C)対に対して、高齢者の悲しみ--落着き(S-C)対、若年健聴者の全実験条件・感情対で有意差があった。
さらに、高齢者の悲しみ--落着き(S-C)対と若年健聴者模擬難聴なしの悲しみ--落着き(S-C)対でも有意差があった。 
このことから、高齢者は若年健聴者と比較して、どの感情対においても判断が難しかったことがわかる。
また、悲しみ--落着き(S-C)対が他の感情対よりも小さくなっており、今回参加した高齢者にとって、悲しみ--落着き(S-C)対の判断が他の感情対の判断よりも容易であったことが示唆された。


%%%%%%%%%%%%%%%%%%%%%%%%%%%%%%%%%
\subsubsection{(b)PSE}
%%%%%%%%%%%%%%%%%%%%%%%%%%%%%%%%%%
図\ref{fig:JNDPSE_Calm}(b)PSEに関して、平均値はすべての条件でおおむね50\%に近かった。
だたし、高齢者の喜び--落着き(H-C)対は若干大きく、若年健聴者模擬難聴ありの怒り--落着き(A-C)対以外のすべての条件との間で有意差があった。
手掛語を区別して検定も行ったが有意差はなく、判断にバイアスが無かったことがわかった。


%%%%%%%%%%%%%%%%%%%%%%%%%%%%%%%%%
\subsubsection{(c)高齢者の聴力レベルによる検討}
%%%%%%%%%%%%%%%%%%%%%%%%%%%%%%%%%%
高齢参加者の聴力レベルによる感情弁別特性の違いを調べた。
高齢参加者を500~Hzから4000~Hzの良耳の平均聴力レベルによって、健聴に近い群(22~dB未満, ONH, 7名)と難聴群(22~dB以上, OHL, 5名)に分けた。
% なお、閾値をWHO基準の25dBに設定した場合でも、結果は変わらなかった。-- 未確認 28Jan2025

図\ref{fig:JNDPSE_Calm}(c)に参加者間のJNDの平均値と95\%信頼区間を示す。
図左側から、若年健聴者模擬難聴なし(YNH-Unpro)、ONH、OHLの結果である。
高齢者は健聴・難聴に関わらず、怒り--落着き(A-C)対、喜び--落着き(H-C)対のJNDが若年健聴者に比べ有意に大きい。
その一方で、悲しみ--落着き(S-C)対のJNDはONH・OHLで同程度であり、他の感情対に比べて小さい。
つまり、高齢者は他の感情対に比べて、悲しみ--落着き(S-C)対の判断がしやすかったことを示唆する。
また、若年健聴者においては、悲しみ--落着き(S-C)対は怒り--落着き(A-C)対よりもJNDが小さいが、平均間の有意差はなかった。

% 以上の結果は、聴力レベルだけではJNDの相違を説明できず、末梢系以降の要因があることを示唆する。



%%%%%%%%%%%%%%%%%%%%% %%%%%%%%%%%%%%%%%%%%% %%%%%%%%%%%%%%%%%%%%% 
% JNDとPSEの統計分析結果(Calm)
%%%%%%%%%%%%%%%%%%%%%% %%%%%%%%%%%%%%%%%%%%% %%%%%%%%%%%%%%%%%%%%% %%%%%%%%%%%%%%%%%%%%% 
\begin{figure}[t]
  % \vspace {-20pt}
  
  
  \begin{tabular}{ccc}
  
  \begin{minipage} {0.32\hsize}
  \centering
  \includegraphics[ width = 1\columnwidth]{Figure/ExpCalm/Fig_YNHEld_JND_IgnoreCue.eps }
  \end{minipage}&
  
  \begin{minipage} {0.32\hsize}
  \centering
  \includegraphics [ width = 1\columnwidth]{Figure/ExpCalm/Fig_YNHEld_PSE_IgnoreCue.eps }
  \end{minipage}&

  \begin{minipage} {0.32\hsize}
  \centering
  \includegraphics [ width = 1\columnwidth]{Figure/ExpCalm/Fig2c_CalEmoYNHEld_JNDNHHL_Mean4_Thrsh22dB.eps }
  \end{minipage}
  

  \end{tabular}
  
  \vspace {-6pt}
  \caption{実験条件(YNH-Unpro,YNH-80yr,Older-Unpro)ごとにおけるJND(a)とPSE(b)の参加者全体の平均値と、95\%信頼区間。
            点線はJND,PSEそれぞれの全条件の平均値を示す。
            (c)はYNH-Unpro、高齢健聴群(ONH)、高齢者難聴群(OHL)のJNDの平均値と95\%信頼区間。
            各3組のバーグラフは感情対に対応。落着き (Calm, C)。
            Tukey HSDの多重比較検定(有意水準 5\%)の結果も示す。}
  
  \label{fig:JNDPSE_Calm}

  % \vspace {-12pt}
  \end{figure}
  %%%%%%%%%%%%%%%%%%%%%% %%%%%%%%%%%%%%%%%%%%% %%%%%%%%%%%%%%%%%%%%% %%%%%%%%%%%%%%%%%%%%% 




%%%%%%%%%%%%%%%%%%%%% %%%%%%%%%%%%%%%%%%%%% %%%%%%%%%%%%%%%%%%%%% 
% JJND Box ONHOHL Calm vs. AngSadHap
%%%%%%%%%%%%%%%%%%%%%% %%%%%%%%%%%%%%%%%%%%% %%%%%%%%%%%%%%%%%%%%% %%%%%%%%%%%%%%%%%%%%% 

\begin{figure*}[t]
  % \vspace{-18pt}
  
  \begin{tabular}{cc}
  % \hspace{-32pt}
  \begin{minipage} {0.5\hsize}
  \centering
  \includegraphics[ width = 1\columnwidth]{Figure/ExpCalm/FigBox_Cal_JNDNHHL_Mean4_Thrsh22dB.eps }
  \end{minipage}&
  \hspace{-22pt}
  
  \begin{minipage} {0.5\hsize}
  \centering
  \includegraphics [ width = 1\columnwidth]{Figure/ExpCalm/FigBox_AngSadHap_JNDNHHL_Mean4_Thrsh22dB.eps }
  \end{minipage} 
  
  
  \end{tabular}
  
  %----------memo----------%
  %(a): CalEmoWHIS24_ScatterCorrJNDvsAgramAge.m実行後,CalEmoWHIS24_StatJND_ONHOHLを実行することで出力される.
  %(b): ExpEmoWHIS24/test_Exp2023/Copy_of_CalEmoYNHEld_ScatterCorrJND実行後,testCalEmoYNHEld_StatJND_ONHOHLを実行することで出力される.
  %------------------------%
  
  \caption{実験条件(YNH-Unpro,YNH-80yr,ONH,OHL)ごとのJND。
          (a)落着き実験、(b)怒り・悲しみ・喜び実験。
           各3組のデータは感情対に対応。落着き(Calm, C), 怒り(Angry, A),悲しみ(Sad, S),喜び(Happy,H)の2つの組み合わせで表示。
           25,50,75パーセンタイルの箱ひげ図で表示。参加者のJND/PSEを手掛語(怒り*, 悲しみ△, 喜び◯, 落着き■)ごとに示す。
          Tukey HSDの多重比較検定(有意水準5\%)の結果も示す。}
  % \vspace{-12pt}
  \label{fig:ExpEmo_BoxPlot} 
  
  %from Scirep24 Fig2 caption
  % (c) shows the mean and 95% CI of the JND to compare YNH-Unpro, older NH (ONH), and older HL (OHL). Tukey’s HSD tests were performed at α = 0.05. In panel (a), there were significant differences between the Hap-Ang pairs for the older participants with asterisks (*) and the other pairs. The dashed line represents the mean of all conditions except the last two bars. In panel (b), the pairs with asterisks (*) were significantly different from the other pairs. In panel (c), there were also significant differences from the Hap-Ang pair in the older NH.
  
  \end{figure*}


\clearpage
%%%%%%%%%%%%%%%%%%%%%%%%%%%%%%%%%
\section{怒り・悲しみ・喜び実験結果(第\ref{chap:ExpAngHapSad}章)との比較}
%%%%%%%%%%%%%%%%%%%%%%%%%%%%%%%%%%
ここでは、怒り・悲しみ・喜び実験結果との比較を行う。
参加者ごとの分布を見るために、参加者それぞれの結果を箱ひげ図とともに表示した。
図\ref{fig:ExpEmo_BoxPlot}(a)に今回の落着き実験のJND、図\ref{fig:ExpEmo_BoxPlot}(b)に怒り・悲しみ・喜び実験のJNDを箱ひげ図で示す。
また、図中に参加者ごとのJNDを手掛語別に示している。
図左側から、若年健聴者模擬難聴なし(YNH-Unpro)、若年健聴者模擬難聴あり(YNH-80yr)、ONH、OHLの結果である。
これらの12条件(YNH-Unpro・YNH-80yr・ONH・OHL、感情対すべて)に関してTukey HSDの多重比較検定(有意水準5\%)を行った。

図\ref{fig:ExpEmo_BoxPlot}(a)において、高齢者は健聴・難聴に関わらず、怒り--落着き(A--C)対、喜び--落着き(H--C)対のJNDが若年健聴者に比べ有意に大きい。
また、OHLの怒り--落着き(A--C)対、喜び--落着き(H--C)対で非常にばらつきが大きいが、悲しみ--落着き(S--C)対は、1名1対分の例外を除いてONHと同等であった。
このことは、聴力レベルに関わらず、高齢者は悲しみ--落着き(S--C)対の判断がしやすかったことを示唆する。

図\ref{fig:ExpEmo_BoxPlot}(b)の怒り・悲しみ・落着き実験においては、第\ref{chap:ExpAngHapSad}章で述べた通り、高齢者の健聴群・難聴群ともに喜び--怒り(H-A)対のJNDが有意に高い。
ここで、図\ref{fig:ExpEmo_BoxPlot}(a)の落着き実験結果と比較すると、
若年健聴者・高齢者に関わらず悲しみ--落着き(S--C)対のJNDは怒り--悲しみ(A--S)対、悲しみ--喜び(S--H)対と同程度かそれ以下である。
以上のことから、若年健聴者・高齢者に関わらず「悲しみ」を含む感情対の判断が他の感情対に比べて容易であることがわかった。

以上の結果は、聴力レベルだけではJNDの相違を説明できず、末梢系以降の要因があることを示唆する。




%%%%%%%%%%%%%%%%%%%%%%%%%%%%%%%%%
\section{考察}
%%%%%%%%%%%%%%%%%%%%%%%%%%%%%%%%%%
本実験で得られた結果から、高齢者における悲しみ--落着き対の弁別のJNDは、他の対よりも小さいことがわかった。
このことは、怒り・悲しみ・喜び実験で高齢者における怒り--喜び対のJNDが大きいことを説明するための作業仮説
「覚醒度が同程度である場合、快--不快の判断が難しい」を反証したことになる。
むしろ、すべての実験を通して、聴力レベルの程度にかかわらず「悲しみ」を含む感情対でJNDが小さいことから、「悲しみ」音声が特徴的であったことが原因である可能性がある。
実際に、用いた男性1名の悲しみの音声は、演技的で印象に残りやすかった。
ここで、図\ref{fig:PCA-Russel_Calm}の主成分分析結果を見ると、「悲しみ」音声のみ10単語が一点に重なっている。
つまり、悲しみの尺度評定にほかの感情の判断が入っておらず、用いた音声が非常に特徴的であったと推測できる。
% このことは、「悲しみ」音声のみ10単語に関して、図\ref{fig:Russel_Calm}(c)の主成分分析結果が一点に重なっていて、悲しみの尺度評定にほかの感情の判断が入らなかったことからも推測できる。

%%%%%%%%%%%%%%%%%%%%%%%%%%%%%%%%%
\subsubsection{高齢者による感情尺度評定と主成分分析}
%%%%%%%%%%%%%%%%%%%%%%%%%%%%%%%%%%
さらなる検討として、今回の落着き音声選別のために実験者3名で行った感情尺度評定(\ref{sec:5Scale_Calm}項を参照)を、
JNDが外れ値だった1名を除いた高齢参加者11名に行ってもらった。
この結果を、図\ref{fig:PCA_Eld}に示す。
実験者3名の結果の図\ref{fig:PCA-Russel_Calm}(c)と類似の配置で、ばらつきは大きいものの、怒りと悲しみの分布の分離性は良い。
悲しみを含む感情対だけが弁別しやすい理由はこれだけではわからなかった。

%%%%%%%%%%%%%%%%%%%%%% %%%%%%%%%%%%%%%%%%%%% %%%%%%%%%%%%%%%%%%%%% 
% 主成分分析結果・Russel  Calm


% ----------------------------------%
% PCA Eld
\begin{figure}[t] 
  % \vspace{-8pt}
   \begin{center} 
     \hspace{-20pt}
         \includegraphics[width = 0.7\columnwidth]{Figure/ExpCalm/FigPCA3dEld_CumAll_cal2_scatter_SndTriangle_Eng_RotateC.eps}
        \vspace{-3pt}
         \caption{高齢者による感情尺度評定の主成分分析 (PCA) 結果. 図\ref{fig:PCA-Russel_Calm}の若年健聴者結果に対応。}
          \label{fig:PCA_Eld}
     \end{center}  
  % \vspace{-13pt}
 \end{figure}
 % ----------------------------------%


% % ---------------------------------------
% \begin{figure}[h]
%   \vspace{40pt}
%   \hspace{20pt}
%   \centering
%   \includegraphics[width=0.8\hsize]{Figure/ExpCalm/FigPCA3d_CumAll_cal2_scatter_SndTriangle_Eng.eps}
%   \caption{感情尺度評定の主成分分析(PCA)結果。各点(x)は、怒り,悲しみ,喜び,落着きの評価を行って得られたPCAスコアを示す。
%             "Ang"怒り,"Sad"悲しみ,"Hap"喜び,"Cal"落着き。 見やすさのため3次元で描画している。}
% \end{figure}
% % --------------------------------------------





% ----------------------------------%
\begin{figure}[h]
  \vspace{-10pt}
  \centerline{\includegraphics[width=0.7\linewidth]{Figure/ExpCalm/Fig_YNHEld_StatAct_KESD.eps}}
  \vspace{0pt}
  \caption{実験者3名を含む若年健聴者7名・高齢者11名による「演技度」の評定の平均値と95\%信頼区間。
           縦軸:評定値。赤:怒り、青:悲しみ、緑:喜び、黄:落着き音声。YNH:若年健聴者、Older:高齢者。
          }
\vspace{-15pt}
\label{fig:StatAct}
\end{figure}
% ----------------------------------%

\clearpage
%%%%%%%%%%%%%%%%%%%%%%%%%%%%%%%%%
\subsubsection{感情音声の演技度の評価}
%%%%%%%%%%%%%%%%%%%%%%%%%%%%%%%%%%
感情尺度評定(\ref{sec:5Scale_Calm}項)を行った際に、音声の演技度についても評定するタスクを入れていた。
% 感情音声に対し、1.非常に演技らしい $\sim$ 5.日常会話的の5段階で評定を行った。
そこで、この評定値について分析を実施した。
また、若年健聴者に関して、実験者3名のみのデータしかなかったため、追加で新規参加者4名(日本人大学生と大学院生)に実施した。
若年健聴者7名、高齢者11名の音声の感情ごとの評定値の平均を図\ref{fig:StatAct}に示す。 
% 結果を分析したところ、若年健聴者(実験者3名)・高齢者ともに他の感情に比べて「悲しみ」音声が有意に演技度が高かった。
多重比較検定(有意水準5\%)の結果、若年健聴者(実験者3名)・高齢者ともに「悲しみ」音声と、若年健聴者の怒りを除く他の感情で有意差があり、
有意に演技度が高かったことがわかった。
これにより、「悲しみ」音声が演技的であったことが裏付けられたが、弁別のしやすさの要因がこれだけであるかはわからない。
% 評価が高齢者の方がPCA結果中心より?控えめに評価している。他の複数感情に値を入れている可能性も。
% YNH-Older間の比較もしたい



全体の結果から、多くの場合で加齢とともに感情弁別が難しくなることがわかった。
これは、これまでの感情認識研究における加齢の影響を示す結果と整合性がある。
例外的に影響が小さかった悲しみ音声に関しては、加齢や難聴に対して頑健であったとも言い換えられるかもしれない。
その特徴量を計算論的に解明できれば、感情音声伝達を支援するシステムの実現に役立てられる可能性がある。





%%%%%%%%%%%%%%%%%%%%%%%%%%%%%%%%%
\section{まとめ}
%%%%%%%%%%%%%%%%%%%%%%%%%%%%%%%%%%
怒り・悲しみ・喜び間の弁別実験結果(第\ref{chap:ExpAngHapSad}章)を受けて、落着き音声とそれらの3感情間との弁別実験を実施した。
その結果、ラッセルの感情円環モデルに基づいた作業仮説「覚醒度が同程度である場合、快-不快の判断が難しい」は反証された。
また、若年健聴者実験では模擬難聴処理の有無でPSEやJNDに有意差は無かった。
このことから、聴覚末梢系の機能低下だけは感情知覚に影響しないことが示された。
これは、怒り・悲しみ・喜び実験の結果と整合性がある。
また、高齢者は健聴・難聴に関わらず若年健聴者よりもJNDが大きく、弁別精度が下がることがわかった。
落着き実験と怒り・悲しみ・喜び実験結果を比較すると、「悲しみ」音声を含む感情対でJNDが小さく、
これは悲しみ音声が有意に演技的で弁別しやすかった可能性が考えられる。

本実験では、特定の男性1名の音声だけを用いた実験結果のため、男女複数話者の音声を用いてより多くのデータを収集する必要がある。
% そのための音声収録は進んでいる。
また、今回対象とした4感情以外の恐怖、驚き、嫌悪など、異なる感情対での弁別実験も必要であろう。




% \begin{figure}[h]
%   \vspace{10pt}
%   \begin{tabular}{cc}
%   \begin{minipage} {0.47\hsize}
%   \centering
%   \includegraphics [ width = 1\columnwidth]{Figure/ExpCalm/Fig_YNHEld_StatAct_KESD.eps}
%   \subcaption{}
%   \end{minipage} & 
  
%   \begin{minipage} {0.47\hsize}
%   \centering
%   \includegraphics [ width = 1\columnwidth]{Figure/ExpCalm/Fig_YNHEld_StatAct_KESD.eps }
%   \subcaption{}
%   \end{minipage}
  
%   \end{tabular}
  
%   \caption{実験者3名を含む若年健聴者7名・高齢者11名による「演技度」の評定の平均値と95\%信頼区間。
%             縦軸:評定値。赤:怒り、青:悲しみ、緑:喜び、黄:落着き音声。YNH:若年健聴者、Older:高齢者。
%           }
%   \label{fig:StatAct} 

% \end{figure}



% 昨年度の怒り・悲しみ・喜び間の弁別実験結果を受けて、落着き音声とそれらの感情間の弁別実験を実施した。
% ラッセルの感情円環モデルに基づいた作業仮説は反証された。
% また、高齢者が健聴か難聴かによらず若年健聴者よりもJNDが大きく、さらに若年健聴者の模擬難聴処理の有無でPSEやJNDに有意差は無かった。
% この結果から、聴覚末梢系の機能低下だけが感情知覚に影響するわけではないこと示された。
% 特定の男性1名の音声だけを用いた実験結果のため、男女複数話者の音声を用いてより多くのデータを収集する必要がある。
% そのための音声収録は進んでいる。

% 高齢者の感情音声知覚特性を調べるために、3感情間のモーフィング音声を用いて、若年健聴者と高齢者を対象に弁別実験を実施した。
% 若年健聴者には通常音声と模擬難聴音を、高齢者には同じ通常音を聞かせ、計3条件間の対比を行った。
% その結果、若年健聴者実験では模擬難聴処理の有無でJND、PSEに有意差は無く、聴覚末梢系の機能低下は感情知覚に影響しない可能性が示唆された。
% 高齢者実験では、喜び--怒りの判断だけが著しく難しいことがわかった。
% すなわち、高齢者はラッセルの感情円環モデルにおいて、「覚醒--沈静」の度合いが同程度の場合、「快--不快」の判断が難しい可能性がある。
% したがって、第4象限に位置するCalm(「落着き」)を加えて弁別実験を行うことが、要因解明の手がかりになるかもしれない。

