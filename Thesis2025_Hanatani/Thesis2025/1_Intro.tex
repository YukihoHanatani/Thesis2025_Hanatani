
%%%%%%%%%%%%%%%%%%%%%%%%%%%%%%%%%%%%%%%%%%%%%%%%%%%%%%%%%%%%%%%%%%%%%%%%%%%%%%%%%%
%%%%%%%%%%%%%%%%%%%%%%%%%%%%%%%%%%%%%%%%%%%%%%%%%%%%%%%%%%%%%%%%%%%%%%%%%%%%%%%%%%
\chapter{はじめに}
\label{chap:Intro}
%%%%%%%%%%%%%%%%%%%%%%%%%%%%%%%%%%%%%%%%%%%%%%%%%%%%%%%%%%%%%%%%%%%%%%%%%%%%%%%%%%
%%%%%%%%%%%%%%%%%%%%%%%%%%%%%%%%%%%%%%%%%%%%%%%%%%%%%%%%%%%%%%%%%%%%%%%%%%%%%%%%%%


%%%%%%%%%%%%%%%%%%%%%%%%%%%%%%%%%%%%%%%%%%%%%%%%%%%%%%%%%%%%%%%%%%%%%%%%
\section{研究背景}
%%%%%%%%%%%%%%%%%%%%%%%%%%%%%%%%%%%%%%%%%%%%%%%%%%%%%%%%%%%%%%%%%%%%%%%%
\label{sec:研究背景}
超高齢社会となった日本では、加齢性難聴(老人性難聴)者の増加が懸念されている。
聴力低下は、音声による対人コミュニケーションを妨げ、生活の質(QOL)の低下を招く。
また、認知症リスクを高める最大要因であることがLancet委員会により報告されている\cite{livingston2020dementia}。
高齢者との円滑なコミュニケーションのためには、音声の言語情報ばかりではなく、感情を含む非言語情報も重要である。
特に感情(情動)は、言語情報と異なり曖昧で、高齢者にどの程度意図した通りに伝達できるかの特性もまだよくわかっていない。
これまでに、加齢により感情認識精度が低下することが報告されている\cite{paulmann2008aging,ben2019age,amorim2021changes}。
特に否定的な感情に関して低下するという報告もある\cite{mill2009age}。
また、従来からの補聴器で音声了解度の改善は行うことができるが、感情伝達特性の向上には貢献しないことが報告されている\cite{goy2018hearing}。
感情知覚機能の低下が、末梢系の機能低下による聴力レベルの要因だけとは考えられてはいない。
しかし、明確にそのことを示す実験データは存在しないようである。
また、末梢系から認知系のどの部位の機能低下によるものかも解明されていない。
そこで、まずは聴覚末梢系の機能低下だけが、感情知覚にどの程度影響するかの調査を試みた。
そこでは、若年健聴者に模擬難聴処理を行った音声を聴かせ、通常音声を聴いた場合との感情知覚の相違を実験的に検討した。
さらに、高齢者の感情知覚特性を、同じ通常音声を用いて同じ手順で測定した。
若年健聴者との相違から知覚特性の情報を得ることを目指した。

% SciRep
% 難聴(HL)を抱える高齢者の数は多くの国で増加している。
%  HLによる対人コミュニケーションの低下は、生活の質(QOL)を低下させるだけでなく、認知症の極めて高い危険因子であることが示されている1。
% 音声は、言語的内容と、話し手の特徴や感情を含む副言語的情報の両方を伝えます。
% 従来の補聴器は、主に音声の明瞭度を向上させるように設計されており、感情的なコミュニケーションを改善するものではない。
%  音声の感情認識2-5とその加齢効果6-14に関する研究は数多く行われており、加齢とともに感情認識・識別能力が低下することが報告されている。
% 補聴器は明瞭度を向上させるにもかかわらず、感情認識を向上させないことが確認されている9。
% 加齢は、蝸牛から中枢神経系に至る聴覚プロセスに影響を及ぼす。
% 一つの研究課題は、加齢に関連した末梢の聴覚障害が感情認知にどの程度影響するのか、また異なる段階でどのような過程が関与するのかということである。
% もう一つの疑問は、感情知覚をサポートする新しい補聴器を開発できるかどうかということである。
% 感情の低下が、認知のみによるのではなく、主に音声の特徴や声の表情の聴覚的表現の劣化によって引き起こされるのであれば、新しい強化アルゴリズムを開発することも可能であろう。

% 【copied 2022】世界に先駆けて超高齢化社会へ突入した日本では,今後も老人性難聴者が増加することは間違いない.
% 聴力低下による対人コミュニケーションの減少は,生活の質(Quality of life:QOL)を下げることに繋がる\cite{lancet}.
% また,老人性難聴は認知症の極めて高い要因であることが指摘されている\cite{AgeHL}.一般的に,難聴の対処には補聴器が挙げられる.
% その一方で,補聴器は難聴者の15\%ほどしか利用されておらず\cite{JPTrack},万人に有効な手立てとはなっていない.今までの補聴器は,
% 主に聞き取りやすさ(音声了解度)の改善だけを目的とされてきた.しかし,難聴者との円滑なコミュニケーションのためには音声了解度だけでなく,
% 感情の伝わりやすさ(感情伝達特性)についても考慮する必要があると考える.
% 老人性難聴者に話しかける際には「大きな声でゆっくりと話す」ことがよいとされている.ところが,これを表層的に捉えて単に大きな声で話しかけると,
% 「叱られている」と高齢者に勘違いされることがある.聴力レベルや言語情報処理から見た従来の理論においては正しいとされる方法であるにもかかわらず,
% 「感情」が伝わらないということは,感情伝達についての知見が不十分な証拠である.

% 本研究の将来的な目標は,老人性難聴者を含む他者への感情伝達特性を解明し,定式化することである.これにより,
% 補聴器の開発や超高齢社会における対人コミュニケーションの改善に繋がると考える.


%%%%%%%%%%%%%%%%%%%%%%%%%%%%%%
\section{研究目的}
%%%%%%%%%%%%%%%%%%%%%%%%%%%%%%
\label{sec:研究目的}

本研究の目的は、高齢者の感情知覚特性について調査することである。
そのために、2つの実験を実施する。

1つ目は、若年健聴者実験である。
若年健聴者に難聴者の聞こえを体験させる「模擬難聴システムWHIS」(以下WHIS)を用いた聴取実験を行い、模擬難聴が感情知覚に与える影響を調査する。
通常音声を聞いた場合と、模擬難聴処理した音声を聞いた場合の結果に差異があるかどうかを確かめる。
WHISを用いて実験を行う理由は以下である。
加齢性難聴の原因は様々であり、聴覚末梢系の機能低下によるものか、中枢系・認知系の機能低下によるものかの切り分けが困難である。
どの原因がどの程度感情知覚に影響するのかを段階的に調査するために、まずは模擬難聴処理で聴覚末梢系の機能低下だけを模擬することにした。
したがって、WHISで処理した音声を聞いた若年健聴者を「模擬難聴者」として実験を行う(詳しくは,第\textcolor{red}{☆☆☆}章で説明する)。

2つ目は、高齢者実験である。
若年健聴者実験と同じ通常音声・手順で実験的に測定を行う。
若年健聴者結果と比較することで、聴覚末梢系の機能低下による感情知覚への影響があるかどうかを検証する。
また、聴覚末梢系の機能低下以外の要因も調査できると考え、実施することにした。

これらの実験から、若年健聴者が通常音声を聞いた場合・若年健聴者が模擬難聴処理音声を聞いた場合・高齢者が通常音声を聞いた場合の
感情音声の弁別性能を比較する。
これらの3条件の違いから、高齢者の感情知覚特性について新たな知見を得ることを目的とした。


%SciRep
% 第一の研究課題に答えるためには、高齢者では変動が大きい聴覚経路や認知の要因を除外し、末梢HLのみがパーフォーマンスに及ぼす影響を特定する必要もある。
% この目的のために、健聴者(NH)にHL体験を提供するHLシミュレータを実験に使用することができる。
% HL模擬音を使った怒りの知覚に関する最近の実験がある14。
% 動機は似ているが、彼らの実験は基本的に感情識別課題であった。

%本研究では、音声モーフィングツールとHLシミュレータWHIS24を組み合わせて、感情ペア間の感情弁別実験を行った。
% 若年NH(YNH)参加者が通常の音を聞いた場合、同じYNH参加者がHL模擬音を聞いた場合、高齢参加者が同じ通常の音を聞いた場合の識別性能を比較した。
% これら3つの条件の違いから、高齢者の感情知覚の特徴について新たな知見が得られるかもしれない。

% 【copied 2022】本研究は,健聴者に難聴者の聞こえを体験させる「模擬難聴システムWHIS」(以下WHIS)を用いた聴取実験を行い,
% 模擬難聴が感情知覚に与える影響を調査することを目的とする.将来的な目標は,老人性難聴における感情伝達特性の解明と定式化である.


% WHISを用いて実験を行う理由は以下の2つである.第一に,実際の老人性難聴者を対象とした場合,難聴の原因が末梢系の機能低下によるものか,
% 認知機能の低下によるものかの切り分けが困難であり,個人ごとのばらつきも大きい.第二に,新型コロナウイルスの影響もあり高齢者を対象とした実験は難しい.
% このような理由から,WHISで処理した音声を聞いた健聴者を「模擬難聴者」として実験を行う(詳しくは,第\ref{sec:感情知覚実験}章で説明する).

% 本研究では,同一被験者で健聴状態と模擬難聴状態で聴取実験を行い,感情判断の差異をもとに模擬難聴が感情知覚に及ぼす影響について調査する.


%%%%%%%%%%%%%
\section{\textcolor{red}{本論文の構成}}
%%%%%%%%%%%%%
\label{sec:本論文の構成}
【copied 2022】本論文は,本章を含め6章で構成されている.第1章では,本研究の背景と目的について示した.第2章では,聴覚に関する基本的な知識と,音声と感情知覚に
関する知識,先行研究について説明する.第3章では,本実験の詳細な手続きを述べる.第4章では実験結果と統計的分析を行なった結果を示し,第5章で考察を述べる.
最後に,第6章にて本研究についてまとめる.

