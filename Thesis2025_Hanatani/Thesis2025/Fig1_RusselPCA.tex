%%%%%%%%%%%%%%%%%%%%%% %%%%%%%%%%%%%%%%%%%%% %%%%%%%%%%%%%%%%%%%%% 
% 主成分分析結果・Russel  Ang-Hap-Sad

\begin{figure}[t]

  \begin{tabular}{cc}
  \begin{minipage} {0.47\hsize}
  \centering
  \includegraphics [ width = 1\columnwidth]{Figure/ExpAngHapSad/FigPCA_CumAll_scatter_SndTriangle_Eng.eps}
  \end{minipage} & 
  
  \begin{minipage} {0.47\hsize}
  \centering
  % \includegraphics [ width = 1\columnwidth]{Figure/ExpAngHapSad/Fig_RusselCircle_b.eps }
  \includegraphics [ width = 1\columnwidth]{Figure/ExpAngHapSad/Fig_RusselCircle_b_JLabel.eps }
  \end{minipage}
  
  \end{tabular}
  
  \caption{感情尺度評定の主成分分析(PCA)結果(a)とラッセルの感情円環モデル(b)(\cite{russell1980circumplex}のFig.4より再描画)。
            図(a)の各点(x)は、単語ごとに基本6感情("Ang"怒り,"Sad"悲しみ,"Hap"喜び,"Fea"恐怖,"Dis"嫌悪,"Sur"驚き)の評価を行って得られたPCAスコアを示す。
            薄い線の三角形は抽出された10単語それぞれを結ぶ。
            ラッセルの円環モデル(b)内の感情との位置関係に合わせるため、横軸をPC2、縦軸をPC1の符号反転としている。
            }
  \label{fig:PCA-Russel_AngHapSad} 

\end{figure}

