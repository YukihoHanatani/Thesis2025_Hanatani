%%%%%%%%%%%%%%%%%%%%%% %%%%%%%%%%%%%%%%%%%%% %%%%%%%%%%%%%%%%%%%%% 
% 主成分分析結果・Russel  Calm


% ----------------------------------%
% PCA Eld
\begin{figure}[t] 
  % \vspace{-8pt}
   \begin{center} 
     \hspace{-20pt}
         \includegraphics[width = 0.7\columnwidth]{Figure/ExpCalm/FigPCA3dEld_CumAll_cal2_scatter_SndTriangle_Eng_RotateC.eps}
        \vspace{-3pt}
         \caption{高齢者による感情尺度評定の主成分分析 (PCA) 結果. 図\ref{fig:PCA-Russel_Calm}の若年健聴者結果に対応。}
          \label{fig:PCA_Eld}
     \end{center}  
  % \vspace{-13pt}
 \end{figure}
 % ----------------------------------%


% % ---------------------------------------
% \begin{figure}[h]
%   \vspace{40pt}
%   \hspace{20pt}
%   \centering
%   \includegraphics[width=0.8\hsize]{Figure/ExpCalm/FigPCA3d_CumAll_cal2_scatter_SndTriangle_Eng.eps}
%   \caption{感情尺度評定の主成分分析(PCA)結果。各点(x)は、怒り,悲しみ,喜び,落着きの評価を行って得られたPCAスコアを示す。
%             "Ang"怒り,"Sad"悲しみ,"Hap"喜び,"Cal"落着き。 見やすさのため3次元で描画している。}
% \end{figure}
% % --------------------------------------------



