
%%%%%%%%%%%%%%%%%%%%% %%%%%%%%%%%%%%%%%%%%% %%%%%%%%%%%%%%%%%%%%% 
% モーフィングの概念図と心理物理曲線(Calm)
%%%%%%%%%%%%%%%%%%%%%% %%%%%%%%%%%%%%%%%%%%% %%%%%%%%%%%%%%%%%%%%% %%%%%%%%%%%%%%%%%%%%% 
\begin{figure}[t]
  % \vspace {-20pt}
  %%%%%%%%%%%%%%%%%%%%% 上段 morphing 図 %%%%%%%%%%%%%%%%%%%%% 
  
  \begin{center}
  
  %\includegraphics [ width = 0.35\columnwidth]{FigPCA_morphing.png}
  \includegraphics [ width = 0.65\columnwidth]{Figure/ExpCalm/FigPCA_morphRatioArrow_Calm_Eng.eps}
  \end{center}
  %\caption{Conceptual diagram of placement of emotional morphing voices. Morphing ratio: ×: 50\%, o: 20,40,60,80\% }
  % \label{fig:ExpEmoWHIS _ang-hap-sad }
  
  % \end{figure}
  %----------------------------------%
  
  % \begin{figure}[t]
  \vspace {-12pt}
  \begin{tabular}{ccc}
  %%%%%%%%%%%%%%%%%%%%% 中段 %%%%%%%%%%%%%%%%%%%%% 
  
  \begin{minipage} {0.31\hsize}
  \centering
  \includegraphics[ width = 1\columnwidth]{Figure/ExpCalm/FigYNH_Raw_AllSbj_cal-ang.eps }
  \end{minipage}&
  
  \begin{minipage} {0.31\hsize}
  \centering
  \includegraphics [ width = 1\columnwidth]{Figure/ExpCalm/FigYNH_Raw_AllSbj_cal-sad.eps }
  
  \end{minipage} &
  
  \begin{minipage} {0.31\hsize}
  \centering
  \includegraphics [ width = 1\columnwidth]{Figure/ExpCalm/FigYNH_Raw_AllSbj_cal-hap.eps }
  
  \end{minipage} 
  
      
  \\  %% 改行  %%%%%%%%%%%%%%%%%%%%% 
  
  %%%%%%%%%%%%%%%%%%%%% 下段 %%%%%%%%%%%%%%%%%%%%% 
  
  
  \begin{minipage} {0.31\hsize}
  \centering
  \includegraphics [ width = 1\columnwidth]{Figure/ExpCalm/FigEld_Raw_AllSbj_cal-ang.eps }
  \end{minipage}&
  
  \begin{minipage} {0.31\hsize}
  \centering
  \includegraphics [ width = 1\columnwidth]{Figure/ExpCalm/FigEld_Raw_AllSbj_cal-sad.eps }
  \end{minipage} &
  
  \begin{minipage} {0.31\hsize}
  \centering
  \includegraphics [ width = 1\columnwidth]{Figure/ExpCalm/FigEld_Raw_AllSbj_cal-hap.eps }
  \end{minipage}
  
  \end{tabular}
  
  \vspace {-6pt}
  % \caption{ Results of the emotion discrimination experiments. The top panel (a) shows a schematic plot of the stimulus sounds with morphing ratios of 50\% (x) and 20\%, 40\%, 60\%, and 80\% (o) between the emotions ``anger'' (Ang), ``sadness'' (Sad), and ``happiness'' (Hap), respectively. The morphing ratio of Ang relative to Hap is shown as an example.
  % The middle panels show the means and standard deviations of the percentage responses across YNH participants for the Ang-Sad pair (b), the Sad-Hap pair (c), and the Hap-Ang pair (d). The bottom panels show those across older participants for the Ang-Sad pair (e), the Sad-Hap pair (f), and the Hap-Ang pair (g). Horizontal axis: Vocal morphing ratio (\%).  Vertical axis: Percent response (\%) of Sad or 100-Ang, Hap or 100-Sad, and Ang or 100-Hap. Line colors correspond to cue words in emotion judgments.
  % }
  \caption{感情弁別実験の刺激音配置と結果。
            上図(a)は、感情「怒り」(Ang)、「悲しみ」(Sad)、「喜び」(Hap)、「落着き」(Cal)のモーフィング率が50\%(x)、20\%、40\%、60\%、80\%(o)の刺激音配置の模式図。
            中段は若年健聴者全体の回答の割合の平均と標準偏差を示していて、ペアが怒-落(b)、悲-落(c)、喜-落(d)の場合である。
            下図は高齢者全体の回答で、怒-落ペア(e)、悲-落ペア(f)、喜-落ペア(g)の場合である。
            横軸: モーフィング率(\%)。 縦軸: Calまたは100-Ang、Calまたは100-Sad、Calまたは100-Hapの回答率。}
  
  \label{fig:ExpRsltEmoPercent_Calm}

  \vspace {-12pt}
  \end{figure}
  %%%%%%%%%%%%%%%%%%%%%% %%%%%%%%%%%%%%%%%%%%% %%%%%%%%%%%%%%%%%%%%% %%%%%%%%%%%%%%%%%%%%% 


