%%%%%%%%%%%%%%%%%%%%%% %%%%%%%%%%%%%%%%%%%%% %%%%%%%%%%%%%%%%%%%%% 
% 主成分分析結果・Russel  Calm

% ---------------------------------------
\begin{figure}[h]
  \vspace{40pt}
  \hspace{20pt}
  \centering
  \includegraphics[width=0.8\hsize]{Figure/ExpCalm/FigPCA3d_CumAll_cal2_scatter_SndTriangle_Eng.eps}
  \caption{感情尺度評定の主成分分析(PCA)結果。各点(x)は、怒り,悲しみ,喜び,落着きの評価を行って得られたPCAスコアを示す。
            "Ang"怒り,"Sad"悲しみ,"Hap"喜び,"Cal"落着き。 見やすさのため3次元で描画している。}
  \label{fig:PCA-Russel_Calm} 
\end{figure}
% --------------------------------------------


% \begin{figure}[t]

%   \begin{tabular}{cc}
%   \begin{minipage} {0.47\hsize}
%   \centering
%   \includegraphics [ width = 1\columnwidth]{Figure/ExpAngHapSad/FigPCA_CumAll_scatter_SndTriangle_Eng.eps}
%   \end{minipage} & 
  
%   \begin{minipage} {0.47\hsize}
%   \centering
%   \includegraphics [ width = 1\columnwidth]{Figure/ExpAngHapSad/Fig_RusselCircle_b.eps }
%   \end{minipage}
  
%   \end{tabular}
  
%   \caption{感情尺度評定の主成分分析(PCA)結果(a)とラッセルの感情円環モデル(b)(\cite{russell1980circumplex}のFig.4より再描画)。
%             図(a)の各点(x)は、単語ごとに基本6感情("Ang"怒り,"Sad"悲しみ,"Hap"喜び,"Fea"恐怖,"Dis"嫌悪,"Sur"驚き)の評価を行って得られたPCAスコアを示す。
%             薄い線の三角形は抽出された10単語それぞれを結ぶ。
%             ラッセルの円環モデル(b)内の感情との位置関係に合わせるため、横軸をPC2、縦軸をPC1の符号反転としている。
%             }
%   \label{fig:PCA-Russel_Calm} 

% \end{figure}

