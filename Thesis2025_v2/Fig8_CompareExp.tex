
%%%%%%%%%%%%%%%%%%%%% %%%%%%%%%%%%%%%%%%%%% %%%%%%%%%%%%%%%%%%%%% 
% JJND Box ONHOHL Calm vs. AngSadHap
%%%%%%%%%%%%%%%%%%%%%% %%%%%%%%%%%%%%%%%%%%% %%%%%%%%%%%%%%%%%%%%% %%%%%%%%%%%%%%%%%%%%% 

\begin{figure*}[t]
  % \vspace{-18pt}
  
  \begin{tabular}{cc}
  % \hspace{-32pt}
  \begin{minipage} {0.5\hsize}
  \centering
  \includegraphics[ width = 1\columnwidth]{Figure/ExpCalm/FigBox_Cal_JNDNHHL_Mean4_Thrsh22dB.eps }
  \end{minipage}&
  \hspace{-22pt}
  
  \begin{minipage} {0.5\hsize}
  \centering
  \includegraphics [ width = 1\columnwidth]{Figure/ExpCalm/FigBox_AngSadHap_JNDNHHL_Mean4_Thrsh22dB.eps }
  \end{minipage} 
  
  
  \end{tabular}
  
  %----------memo----------%
  %(a): CalEmoWHIS24_ScatterCorrJNDvsAgramAge.m実行後,CalEmoWHIS24_StatJND_ONHOHLを実行することで出力される.
  %(b): ExpEmoWHIS24/test_Exp2023/Copy_of_CalEmoYNHEld_ScatterCorrJND実行後,testCalEmoYNHEld_StatJND_ONHOHLを実行することで出力される.
  %------------------------%
  
  \caption{実験条件(YNH-Unpro,YNH-80yr,ONH,OHL)ごとのJND。
          \textcolor{red}{(a)落着き実験、(b)怒り・悲しみ・喜び実験。}
           各3組のデータは感情対に対応。落着き(Calm, C), 怒り(Angry, A),悲しみ(Sad, S),喜び(Happy,H)の2つの組み合わせで表示。
           25,50,75パーセンタイルの箱ひげ図で表示。参加者のJND/PSEを手掛語(怒り*, 悲しみ△, 喜び◯, 落着き■)ごとに示す。
          Tukey HSDの多重比較検定(有意水準5\%)の結果も示す。}
  % \vspace{-12pt}
  \label{fig:ExpEmo_BoxPlot} 
  
  %from Scirep24 Fig2 caption
  % (c) shows the mean and 95% CI of the JND to compare YNH-Unpro, older NH (ONH), and older HL (OHL). Tukey’s HSD tests were performed at α = 0.05. In panel (a), there were significant differences between the Hap-Ang pairs for the older participants with asterisks (*) and the other pairs. The dashed line represents the mean of all conditions except the last two bars. In panel (b), the pairs with asterisks (*) were significantly different from the other pairs. In panel (c), there were also significant differences from the Hap-Ang pair in the older NH.
  
  \end{figure*}